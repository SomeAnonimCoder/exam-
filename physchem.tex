\documentclass[11pt]{article}
\usepackage[utf8]{inputenc}
\usepackage[russian]{babel}
\title{Физическая Химия\\\emph{1 курс, второй модуль}}
\author{Андрей  Борисович Ярославцев}
\date{}
\begin{document}
\begin{titlepage}
\maketitle
\end{titlepage}
\tableofcontents
\section{Лекция 1. Основные понятия химической термодинамики. Первый закон термодинамики}

Причина взаимодействий в химии - разность энергий связей. Например, $E_{H-H} + E_{Cl-Cl} < 2E_{H-Cl}$, поэтому реакция $$H_2 + Cl_2 \rightarrow 2HCl$$ идет. У фтора разность больше, реакция идет быстрее, у брома и иода меньше, реакция медленнее. Для всех этих рассуждений нам необходимо было понятие энергии связи.

Знания физхимии нужны для того, чтобы понимать закономерности, читать спец. литературу, понимать другие химические дисциплины.

Матаппарат: 
\begin{itemize}
\item дифференциальное исчисление
\item интегральное исчисление
\item некоторые газовые законы
\end{itemize}

Этапы становления ФХ

\begin{itemize}
\item Адсорбция газов
\item Гальванические элементы, электролиз
\item Теплота реакции
\item Катализ
\item I и II закон термодинамики
\item Термодинамические аспекты химического равновесия
\end{itemize}

В XX веке развивались исследования в области строения молекул, кристаллов, приборных методов анализа, химической кинетики, термодинамики неравновесных  процессов, визуализация все более мелких - вплоть до атомов - частиц, нанотехнологий. 

Физика - наука про энергию  и ее превращения. Химия - наука о веществах и их превращениях. Физхимия - внятного определения нет.

\textbf{Хорошо, когда вы понимаете быстро, но лучше когда вы понимаете правильно \copyright Какой-то там университет}

Разделы курса:

\begin{itemize}
\item Химическая термодинамика
\item Фазовые и химические равновесия
\item Фазовые диаграммы
\item Строение растворов и процессы в них
\item Очистка химических  веществ
\item ...
\end{itemize}

\textbf{Весь контроль - неблокирующий.}

Книжки - 100500 вариантов, советуется книга за авторстовом А.Б.

\subsection{Основные понятия химической термодинамики}

\emph{Хим. термодинамика - изучает превращение химической энергии в теплоту и работу.}

Объекты изучения - балансы химических процессов, фазовые и химические равновесия.

\emph{Под  ТД системой подразумевается некоторая часть пространства со всеми включенными в нее компонентами, являющаяся объектом рассмотрения.}

Термодинамика рассматривает только макроскопические свойства системы. Если частиц в системе мало - её свойства предсказуемы, и \textbf{она не ТД система}. Вся термодинамика - это статистика.

Все положения ТД основаны на ряде постулатов, и рассматривают обобщенные случаи, пользуясь лишь основными законами природы. Благодаря этому, термодинамика очень консервативная наука, <<устойчивая>> к новым открытиям. Эйнштейн писал, что это единственная наука, результаты которой никогда не будут пересмотрены.

Всякая ТД система должна быть ограничена \textbf{поверхностью раздела} - некоторой воображаемой или реальной границей. Через нее может осуществляться обмен различными формами энергии. 

Окружающая среда - все, что вокруг нас, оно принято бесконечным, следовательно, никакие действия системы не могут на него повлиять.
Типы ТД систем:
\begin{itemize}
\item Если обмен и веществом,  и энергией разрешен - система открытая. Пример - чашка кофе, человек.

\item Если обмен только энергией - система замкнутая, или закрытая. Пример - воздушный шарик, запаянная ампула,

\item Если обмен только веществом - система адиабатическая. Пример - сосуд дьюара, калориметр.

\item Если же запрещены все типы взаимодействия - система изолированная. Пример - термос, закрытый герметично и имеющий идеальную изоляцию.
\end{itemize}
ТД состояние системы - совокупность свойств этой системы. Любое изменение состояния - ТД процесс. Для полного описания системы достаточно лишь некоторого количесва свойств. Эти свойства - \textbf{параметры состояния}.

$$pV = \frac{m}{M}RT$$
Параметров два: третья зависит от двух. Это могут быть $(p,V)$, $(P, T)$, $(V, T)$.

Параметры бывают интенсивные и экстенсивные. Экстенсивные - зависят от количества вещества. Аддитивны. Интенсивные параметры - не зависят от количества вещества. Например, давление, напряженность поля, сила. Измеряются только через связанные с ними экстенсивные параметры - например, температуру мы измеряем через длинну столбика ртути.

Любая работа может быть представлена произведением ее интенсивного параметра на изменение интенсивного. $$dA = pdV$$

Удельные значения экстенсивных параметров - удельный объем, концентрация, и т.д. - тоже интенсивные. 

\emph{Равновесное состояние } - такое состояние, при котором свойства системы не меняются от времени, и в ней нет потоков вещества и энергии. На самом деле, в равновесной системе в поле действия внешних сил интенсивные параметры могуут меняться в пространстве. Пример тому - атмосфера земли.

\subsection{Нулевой закон термодинамики}

\emph{Если на границе системы с окружающей средой поддерживаются постоянные значения интенсивных параметров, то эта система рано или поздно приходит в равновесное состояние.}

Равновесен ли слиток олова - интуитивно да, а вот нет, равновесная форма - серое олово.

\subsection{Первый закон термодинамики}

Химические реакции сопровождаются выделением или поглощением энергии. 

В качестве единицы энергии выбраны килокалории(ккал) или килоджоули(кДж). Удельный тепловой эффект - в них же на моль. Сейчас принята система СИ, так что калории - несистемная единица.

Уравнения с тепловым эффектом пишут или до минимальных целых  коэффициентов, или до 1 моля того вещества, которое характеризуем.
$$2H_2 + O_2 \Rightarrow 2H_2O + 484 kJ$$
$$H_2 + \frac{1}{2}O_2 \Rightarrow H_2O + 242 kJ$$

Тепловой эффект - определяется энергией связи в реагентах и продуктах, а теплота и произведенная работа - зависят от пути процесса.

Внутренняя энергия - функция состояния. Это потенциальный запас энергии, состоящий из энергии взаимодействий атомов

Первый закон термодинамики:
\emph{В ходе любого процесса приращение внутренней энергии равно разности между количеством сообщенной ей теплоты и совершенной работы.}
$$\Delta U = Q - A$$

$NB!$ Все в термодинамике пишется с точки зрения системы. Плюс - система приобрела энергию. Минус - отдала.

Для описания процессов при постоянном давлении используют \emph{энтальпию}:
$$H = U+pV = U+\nu RT$$
Энтальпия - функция состояния, как и внутренняя энергия.

Если изменений объема нет, то $\Delta U = \Delta H$

Применение энтальпии логично, если совершается работа по расширению/сжатию против внешних сил. 

$$\Delta U + \Delta (pV) = \Delta H$$

Отсюда ещё одна форма записи тд уравнения - обычное уравнение с подписью $\Delta H = ...$.

Работа при изотермическом расширении идеального газа равна

$$A = \int pdV = \nu RT \int \frac{dV}{V} = \nu RT ln\frac{V_2}{V_1} = \nu RT ln \frac{p1}{p2}$$

Так как состояние системы зависит от условий - давления и температуры, вводят понятие стандартных  условий. Это $298 K$, 1 атм. 

$NB!$ Температура в ТД - всегда в кельвинах.

\section{Термохимия. Закон Гесса}
$$P = const \Rightarrow  Q = \Delta U$$
$$V = const \Rightarrow Q = \Delta H$$

Процессы с выделением тепла, то есть отрицательным $\Delta U$ или $\Delta H$ - экзотермические. Набоборот - эндотермические.

\subsection{Закон Гесса} \emph{тепловой эффект химической реакции зависит только от природы и состояния исходных и конечных веществ, но не от путей протекания реакций и состояния промежуточных продуктов}

Закон гесса позволяет определять теплоту некоторых реакций, например реакций образования $CO$ или $FeO$. Например, запишем цикл: 
$$C+O_2 \Rightarrow CO_2, \Delta H_1$$
$$C+ 1/2 O_2 \Rightarrow CO , \Delta H_2$$
$$CO + 1/2O_2 \Rightarrow CO_2 \Delta H_3$$

По закону Гесса, $\Delta H_1 = \Delta H_2 +\Delta H_3$, откуда несложно найти   $\Delta H_2$, а найти $\Delta H_1$ и $\Delta H_3$ довольно легко экспериментально.

\subsection{Энтальпия образования}

Представив, что продукты и реагенты образовались из простых веществ в стандартном состоянии, получим, что 
$$k_1R_1+k_2R_2+\ldots \Rightarrow n1P_1 + n_2P_2 + \ldots$$
$$\Delta_rH = \sum \Delta_fH_{prod} - \sum \Delta_fH_{reag} = \sum k_i\Delta_fH_i - \sum n_j\Delta_fH_j$$

Аналогично можно записать и для энтальпий сгорания, только будет вычитаться реагенты из продуктов. При этом учитывается сгорание до максимальной с.о.

$$\Delta_rH = \sum \Delta_bH_{reag} - \sum \Delta_bH_{prod}$$

\subsection{Энергия связи}

Через энергию связи считать может быть удобнее, но проблема в том, что кратность связи не дает ее энергию со стопроцентной точнстью, так как энергия связи $CO$, например в $CO_2$ и   $C_xH_y-C(O)-C_xH_y$, сильно различается. А уж у какого-нибудь нитробензола, вообще, все $C-H$ связи имеют разную энергию. Таким образом, считать таким образом можно, но это будет именно оценка, а не точный рассчет, в отличие от других вариантов. 

$$C_T = \frac{dQ}{dT}$$

$$\Delta Q = \int C_TdT$$

$$C_{cp} = \frac {\Delta Q}{\Delta T}$$

$$C_V = \frac{dU}{dT}_V/n$$

$$C_p = \frac{dH}{dT}_p/n$$

У твердых тел и жидкотей,$C_P = C_V$

$$C_p = \frac{dH}{dT} = \frac{dU}{dT} + R = C_v + R$$


по молекулярно-кинетической теории, 

$$E = 3/2kT$$

$$C_v = 3/2R, C_p = 5/2R$$

Это справедливо только для идеального (одноатомного) газа.

\subsection{Фазовые переходы}

Если график температуры от времени плавный, то фазовых переходов на нем нет. Если же на нем есть плато, то это говорит о наличии фазового перехода.

Фазовый переход I рода -превращение, при котором меняются его ТД параметры, такие как энергия Гиббса, энтальпия, теплоемкость.

Фазовый переход II рода - превращение, в ходе которого на кривой теплоемкости появляется разрыв, а на всех других - излом.

Все переходы II рода - это переходы порядок-беспорядок, но обратное неверно.

\subsection{Закон Кирхгоффа}

$$k_1R_1+k_2R_2+\ldots \Rightarrow n1P_1 + n_2P_2 + \ldots$$

$$\Delta H_{T_2} = \Delta H_{T_1} + \int _{T1} ^{T_2}\Delta C_p dT + \sum n_j\Delta H_{P_j} - \sum k_j(\Delta H_{R_j})$$

Если $\sum C_p$ и $\sum C_r$ пересекаются на графике - есть экстремум в $\Delta_rH$. 

\subsection{Второй закон термодинамики}
\textbf{Равновесный процесс} - такой процесс, в любой фазе которого все части системы находятся в равновесии между собой и с окружпющей средой.

Обратимый процесс сопроождается постоянной величиной работы, изменения внутренней энергии, а соответственно и изменением теплоты.

Первый закон ТД говорит о тепловом эффекте, но не о возможности протекания.

Изменение энергии для системы плюс окр. среды равно нулю по ЗСЭ. Совокупная энергия сохраняется, все по I закону ТД. Чтобы разобраться в возможности самопроизвольного протекания процессов, вводят II закон ТД.

\textbf{Существуют процессы, не противоречащие I закону ТД, которые самопроизвольно протекать не могут.}

Другие формулировки:

\textbf{Теплота не может самопроизвольно переходить от холодного тела к горячему.}

\textbf{Теплота более холодного из участвующих тел не может быть источником работы.}

Но отрицательный процесс может протекать, если параллельно с ним протекает другой, не отрицательный. Пример - холодильник.

Для характеристики способности системы к самопроизвольному процессу, вводится энтропия. При обратимом процессе, протекающем через равновесные состояния, энтропия выражается так:
$$\Delta S = S_2 = S_1 = \frac QT$$

Выбор обратимого процесса - логичен, так как в этом случае работа - функция состояния. 

Для необратимого процесса $\Delta S >0$. 

В изолированной систме самопроизвольно могут протекать только процессы, у которых $\Delta S<0$. Это еще одна формулировка II закона ТД.

Единица измерения энтропии - Джоуль на моль-кельвин($\frac J{mol\cdot K}$).

Рассмотрим бильярдный шар, катящийся по столу. Его энергия движения в одну постепенно переходит в энергию движения частиц среды в разные стороны.

Введем понятие ТД вероятности, характеризующей, сколькими микросостояниями может быть реализовано то макросостояние, которое мы наблюдаем.

\textbf{Любая система стремится к наиболее вероятному состоянию.} 

В связи с этим, логично связать энтропию с вероятностью:
$$S=k\ln W$$
где $k$ - постоянная Больцмана

$$\Delta S = k \ln \frac{V_2}{V_1}^n = kN\ln \frac{V_2}{V_1}$$

$$\Delta S = \frac{Q}{T} = R\ln\frac{V_2}{V_1}$$
Из приведенного выше следует, что 
$$R=kN \Rightarrow k=\frac RN$$

Вывод: энтропия - мера разупорядоченности в системе, а II закон ТД постулирует стремление системы к беспорядку.

Каждому телу, веществу и материалу можно приписать определенную энтропию; наличие в веществе примесей увеличивает энтропию.

\textbf{Постулат Планка}

В идеальном кристалле при $0 K$ энтропия равна нулю, а ТД вероятность - единице.

У энтальпии, в отличие от энтропии, нуля отсчета нет, в связи с этим втыкаются костыли типа "энтальпия простых в-в равна нулю"

Существуют эмпирические правила:
\begin{itemize}
\item Правило Труттона: для слабо ассоциированных жидкостей энтропия испарения при температуре кипения примерно равна 90 Дж/моль К. Исключения: уксусная к-та в газе - димер, т.о. изменение энтропии падает($\Delta S =63$ Дж/моль К); вода, напротив, ассоциирована в жидкости, ее изменение энтропии сильно выше - 119 Дж/моль К. 
\end{itemize}

Изменение энтропии для неравновесного процесса, с одной стороны, не имеет смысла, с другой, его можно представить как обратимый с тем же исходным и конечным состоянием, т.о. можно посчитать его энтропию.

\section{Энергия Гиббса и Гельмгольца}

$$\Delta S _{isol} = \Delta S - \frac{\Delta H}{T}$$

Энтальпия системы связана с энтропией среды. Изменение энтальпии это изменение энтропии среды. На самом деле, условие самопроизвольности таково:

$$\Delta S - \frac{\Delta H}{T} >0$$

При малых $\Delta H$ и высокой температуре - доминирует энтропия, при больших $\Delta H$ и малой температуре - доминирует энтальпия.
\subsection{Энегрия Гиббса $G$}
Рассмотрим систему, где происходит работа расширения и работа хим. процесса.

$$\Delta U = Q - p\Delta V - A_{chem}$$
$$-A_{chem} = \Delta U -T\Delta S + p\Delta V = $$

$ \Delta G=-A_{chem}$  - энергия Гиббса

$$\Delta G = \Delta H - T \Delta S $$

\subsection{Энергия Гельмгольца $F$} - то же что и $G$, но для постоянного объема.
$$F = U - TS$$

Процессы всегда протекают в сторону уменьшения энергии Гиббса. Если $\frac {dG}{d\chi} = 0$, то любое изменение состава станет термодинамически невыгодным, то есть у реакции есть положение равновесия, отклонение от которого невыгодно.

Энергия Гемгольца по сравнению с энергией Гиббса бесполезна.

\subsection{Химический потенциал}
Любую работу можно записать через 2 параметра - интенсивный и экстенсиный. Химический потенциал $\mu$ - характеризует скорость изменения энергии Гиббса при изменении ко-ва вещества:
$$\mu = \frac{dG}{dn_i}$$

$$\frac{dG}{dT} = V$$
$$G(p_2) = G(p_1) + \int Vdp = G(p_1) + nRT\ln\frac{p_1}{p_2}$$

Для реальных  газов используют не давление, а летучесть $f$.

Можно перейти и концентрации, что удобно для растворов:
$$\mu = \mu^0 + RT \ln C$$

Для реальных растворов:
$$\mu = \mu^0 + RT \ln a = \mu^0 + RT \ln (\gamma C)$$
,где $a$ - активность, а $\gamma$ - коэффициент активности.

\subsection{Изотерма Вант-Гоффа}
Для реакции 
$$c_1R_1 + c_2R_2 +... = c_1P_1 + c_2P_2 +...$$
$$\Delta G = \Delta G^0 + RT \ln \frac{\sum P_i^{c_i}}{\sum R_j^{c_j}} = \Delta G^0 + RT \ln K$$





\section{Правило фаз Гиббса}

\section{Фазовые диаграммы}

...Сорри, часть материала успешно продолбана...

\subsection{Конгруэнтно плавящиеся соединения.}
Если соединение при плавлении не разлагается, то говорят, что оно плавится конгруэнтно, иначе - неконгруэнтно.

В системе, например, $CuCl\cdot FeCl_3$, уже не простая эвтектика, а по сути, две таких эвтектики: $FeCl_3/CuCl\cdot FeCl_3$ и $CuCl/CuCl\cdot FeCl_3$. В итоге имеем экстремум в середине диаграммы и можем <<разрезать>> ее пополам на две простых эвтектики.

Многие соединения имеют \emph{область гомогенности} - место, где в соединении растворяется один из его компонентов. Пример: $$Fe/FeO/O_2$$. В реальной жизни, состав $FeO$ вовсе даже $Fe_{0.95}O$, что обусловлено переходом малой части $Fe^{2+}$ в $Fe^{3+}$

\subsection{Неконгруэнтно плавящееся соединение}
Возникает дополнительный ФП, в остальном все так же как и раньше
\subsection{Неограниченная взаимная растворимость}
Пример - $Ag/Au$. Диаграмма состояния - "лепесток"; внутри лепестка - 1 степень свободы, на границе - одна, снаружи - две.

\section{Растворы}

Зависимость растворимости от темературы - граница между твердой фазой и р-ром. Интересен вид этой кривой - она имеет экстремум - излом, соответствующий какому-то гидрату вещества, их может быть и несколько

\subsection{Давление над раствором двух неограниченно смешивающихся жидкостей}

Закон Рауля

$$P_a^0 - P_a = P_a^0(1-x_{solvent}) = P_a^0\cdot x_s$$

Добавление в систему компонента меняет ее температуры кипения и плавления в сторону увеличения интервала устойчивости жидкой фазы.

Криоскопическая константа
$$\Delta T_{melt} = Km$$
$$K = \frac{R T_{melt.solv}^2}{1000\Delta H_{melt}}$$

Эбулиоскопическая константа

$$\Delta T_{vap} = Km$$
$$K = \frac{RT^2}{1000\Delta H_vap}$$

\subsection{Осмотическое давление}
Уравнение Вант-Гоффа($\pi$ - осмотическое давление, $C$ - молярная концентрация раствора):
$$\pi = CRT$$

Забавный факт: уравнение весьма похоже на другое:

$$pV = \nu RT$$

Осмос - важен в науке, технике и жизни - вспоминаем наши клетки, протонный шрадиент в митохондриях, фильтры и соленые орурчики.

Степень диссоциации $\alpha$ - доля молекул, распавшихся на ионы. Изотонический коэффициент $i$ - показывает на сколько частиц распадается одна частица вещества в растворе($i_{Al_2(SO_4)_3} =5, i_{AcOH} = 1,\ldots$).

Сильные электролиты - все соли, большинство минеральных кислот - имеют $\alpha > 50\%$

Слабые электролиты - $\alpha < 10\%$ - некоторые соли($ZnCl_2, CdI_2$)

Средние электролиты - $\alpha \in [10 \%;50 \%]$ - щавелевая кислота, фофорная кислота

Закон разбавления Оствальда 

$$K = \frac{C\alpha^2}{1-\alpha} $$

$$\alpha = \sqrt{\frac{K}{C}}$$

\subsection{Произведение растворимости}

Для малорастворимых веществ растворимость определяется произведением растворимости (ПР или $K_s$). Это по сути константа равновесия реакции растворения того самого вещества. Если $K_z <<1$, то вещество плохо растворимо.

$$K_{s_{X_mY_n}} = [X^{n+}]^m\cdot[Y^{m-}]^n$$

\subsection{Что такое кислоты}

По Аррениусу: кислоты - то, что отщепляет протон. Основания - то, что присоединяет:

$$A \Leftrightarrow B^- + H^+$$

По Льюису: кислоты - присоединяют электронную пару, основания - дают. Минусы этой теории - она не предсказывает силу кислот и плохо работает для <<классических>> кислот типа $HCl$. 

Теория Пирсона - жесткие и мягкие кислоты и основания. Жесткие - большая плотность заряда на малом размере, малодеформированные электронные структуры. Мягкие - напротив, легко поляризуемы, как правило большие атомы с низкой ЭО. Примеры: $B^{3+}, Ti^{4+}, O^{2-}, F^{-}$ - жесткие, $Ag^+, Tl^+, J^-, S^{2-}$ - мягкие. Жесткое предпочитает жесткое, мягкое - мягкое.

Количественная трактовка есть, но упоминать ее мы не хотим

\subsection{Автопротолиз и гидролиз}

Рассмотрим растворитель $HA$:

$$2HA \Leftrightarrow A-H---A-H \Leftrightarrow A---H-A-H \Leftrightarrow H_2A^+ + A^-$$


$$\Delta H_{hydrolysis}>0, \Delta S >0$$

$$AB + H_2O \Leftrightarrow A^+ + BH + OH^-$$
$$K_h = \frac{[HB][OH-]}{[AB]}$$

\subsection{Буферные растворы}


\subsection{Методы разделения веществ}

\subsubsection{Экстракция}

Вещества разделяются через разную растворимость в разных растворителях. Пример - экстракция иода неполярными растворителями. из водного раствора

\subsubsection{Транспотртные реакции}
Образование и ращложение легколетучего вещества.

Очистка $Ni$ через образование $Ni(CO)_4$, и затем его разложение. Аналогично с $Zr$ через его иодид.

Условие протекания:
\begin{itemize}
\item $\Delta H$ и $\Delta S$ имеют одинаковый знак
\item $T_1<\frac{\Delta H}{S}<T_2$
\item Давление при $T_0$ достаточно велико
\end{itemize}

\subsubsection{Сорбция и хроматография}

Эти методы разделения основаны на разном сродстве веществ к разным фазам. Пример использования - хроматографическое разделение, для анализа или количественное - для собственно очистки.

\subsection{Мембранные методы}

Диализ - прохождение веществ сквозь мембрану под действием электрического тока. Использется для получения чистой воды, кислот и щелочй. Можно сделать наоборот и концентировать раствор таким образом.

Можно проводить очистку обратным осмосом - это по сути обращение процесса осмоса, когда за счет внешнего давления происходит транспорт вещества против его градиента концентрации.

\subsection{ЭДС реакции}

$$\Delta G = n F E$$

Уравнение Нернста

$$E = E^0 + RT knK$$

Стандартным принят потенциал водородного электрода($H_2|Pt|1M H_2SO_4$). Концентрация серной кислоты в растворе 1M, но $pH=0$, так как есть активность <1, и есть тот факт, что серная кислота не является сильной по второй ступени.

На практике используют хлорсеребрянный электрод - $Ag+AgCl|KCl$, или каломельный $Hg_2Cl_2+Hg|KCl$

\subsection{Направление химических процессов}
$$A+B \Rightarrow C+D$$

$$E = E_{B/D} - E_{C/A} = -\frac{\Delta G}{nF}$$
Реакция идет в прямом направлении, если:
$$E_{B/D} - E_{C/A}>0 $$

Надо учитавыть, что все эти <<идет-не идет>> - штука термодинамическая, то есть если потенциалы говорят, что идет, кинетика может все испортить. Например, алюминий должен бы окисляться воздухом, но нет. 

\subsection{Диаграмма Латимера}

Если известны  $E_{A^{Z_1+}/A^{Z_2+}}$ и $E_{A^{Z_2+}/A^{Z_3+}}$, то $$E_{A^{Z_1+}/A^{Z_3+}} = E_{A^{Z_1+}/A^{Z_2+}} + E_{A^{Z_2+}/A^{Z_3+}}$$ по закону Гесса

Диаграмма Латимера - записанные в ряд по С.О. формы существования эл-та, и потенциалы переходов.

Диаграмма Фроста - смотри прошлый модуль. Они нагляднее и лучше чем диаграммы Латимера, так как позволяют делать выводы о дис- или сопропорционировании.

\subsection{pH-зависимые процессы}

Процессы, в которых участвует $H^+$, имеют потенциалы, зависимые от pH. Пример такого процесса - окисление перманганат-иона. В кислой среде перманганат окисляет хлорид в хлор, а вот в щелочной среде - все наоборот. Это связано с тем, что окисление $Cl^-$ имеет потенциал, не зависящий от pH, а вот $MnO_4$ - очень даже зависящий. 



\section{Кинетика}

Скорость реакции - $\frac{\Delta C}{\Delta t}$ для относительной скорости или в диференциалах для абсолютной.

Нужно учитывать, что скорость по разным реагентам имеет разные значения. В связи с этим используют приведенную скорость - со знаком плюс.

Для большинства процессов, условие протекания - соударение молекул, а скорость процесса пропорциональна частоте соударений. 

$<V> = \sqrt{\frac{3kT}{m}}$

\subsection{Cхема <<циллиндра столкновений>>}
Рассмотрев циллиндр, с радиусом равным 
$H = Vdt$, $S = \pi d^2$, тогда $N = cNSVdt$, $\nu = cNSV$.

Основной постулат кинетики:

Скорость реакции пропорциональна концентрации реагентов в степени их стехиометрических коэффициентов.


\subsection{Элементарные и не очень реакции}

Элементарная реакция - протекает в 1 шаг, для нее работает основной постулат кинетики, и в ней нет переходных состояний. 

Реальные реакции чаще всего состоят из несокльких элементарных стадий. Коэффициенты более 3 - почти наверняка не элементарная реакция.

Реакции второго порядка - и бимолекулярные - достаточно распространены; реакции третьего порядка - уже редки, а уж тримолекулярные - вообще редки.

Кстати, порядок реацкии и молекулярность совпадать не обязаны.

\subsection{Вывод кинетического уравнения}

\begin{center}1 порядок \end{center}

$$\frac{dc}{dt} = k$$
$$\int dc = \int{kdt}$$
$$c = kt + C^0$$
$$C^0 = c(0) = C_o$$
$$c = c_0 + kt$$

\begin{center}2 порядок\end{center}


$$\frac{dc}{dt} = kc$$
$$\int \frac{dc}c = \int{kdt}$$
$$ln c = kt + C^0$$
$$C^0 = ln c(0) =  ln c_o$$
$$c = c_0 \cdot e^{-kt}$$

\subsection{Равновесие}

$$W_1 = W_2$$
$$k_1 A^aB^b = k_2 C^cD^d$$
$$K_{eq} =\frac{ k_1 }{k_2}$$
Аккуратно, не совсем правда, так как работает толькко для элементарных реакций!


\subsection{Параллельные реакции}

Реакции, идущие из одних реагентов по разным направлениям

$$KClO3 \Rightarrow KCl + \frac 32 O_2$$

$$4KClO_3 \Rightarrow KCl + 3KClO_4$$


\subsection{Сопряженные процессы}

Существование одного процесса невозможно без второго. Реакция ниже не идет:

$$H_2O_2 + HJ \Rightarrow $$

Но эта реакция идет при добавлении железа - оно окисляет $HJ$ в $J_2$, восстанавливась в $Fe^{2+}$ затем само окисляется обратно в $Fe^{3+}$ пероксидом, формально не расходуясь и являясь катализатором

\subsection{Метод конкурирующих реакций}
 Пригоден для очень быстрых процессов
 
 $$A + B \Rightarrow C$$
$$A + D \Rightarrow F$$

$$\frac{C}{F} = \frac{k_1B}{k_2D}$$

\subsection{Теория активированного комплекса}

Уравнение Аррениуса

$$k = k_0 +A\exp{\frac{-E_A}{RT}}$$

Правило Вант-Гоффа

При увеличении температуры на 10 градусов, скорость большинства реакций растет в 2..4 раза. 

$$v = v_0*\gamma^{\frac{\Delta T}{10}}, \gamma \in [2..4]$$

Стерический фактор - показывает влияние формы молекул. Пример - молекулы $HJ$ могут столкнуться разными сторонами, и лишь немногие столкновения могут быть удачными.

Энергия активации - по сути, энергия, необходимая для образования активированного комплекса. Складывается из энергии разрыва связи, энергии образования заряженных ионов, если оно есть, и т.д.

Образование радикалов облегчается при наличии молекул имющих к ним сродство

$$RH \Rightarrow R* + H*, 400 kJ$$
$$RH + O_2 \Rightarrow R* + *HO_2, 200 kJ$$

Реакции между радикалами практически всегда безактивационны или имеют очень низкие энергии активации. Концентрация радикалов в реакционной среде не может быть велика из-за этого эффекта.

\subsection{Последовательные реакции}

$$ZrCl_4 + Na \Rightarrow ZrCl_3 \Rightarrow ZrCl_2 \Rightarrow ZrCl \Rightarrow Zr  $$


\subsection{Квазиравновесное и квазистационарное приближения}

Квазистационарное - предполагаем, что скорости орбазования и распада интермедиата равны

Квазиравновесное - предполагаем, что во всех стадиях достигнуто равновесие.

Первое приближение точнее в общем случае, чат как предполагает более реалистичное условие.

Модель Линдемана

$$A + A \Leftrightarrow A* + A (k_2, k_{-2})$$
$$A* \Rightarrow B$$
$$k_2 A^2 = k_1A^* + k_{-2}AA^*$$
$$\frac{dB}{dt} = \frac{k_1k_2}{k_{-2}}A$$






\end{document}





























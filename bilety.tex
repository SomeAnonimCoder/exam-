\documentclass[14pt,a4paper]{scrartcl}
\renewcommand{\sfdefault}{cmr}

\usepackage[utf8]{inputenc}
\usepackage[english,russian]{babel}

\usepackage{indentfirst}
\usepackage{graphicx}
\usepackage{misccorr}
\usepackage{amsmath}
\usepackage{amssymb}
\usepackage{amsfonts}
\usepackage{icomma}
\usepackage{alltt}
\usepackage{enumitem}
\usepackage{soul}
\usepackage{soulutf8}


\begin{document}
	\begin{flushright}
	\textbf{Билеты по физической химии\\
		Залецкая Евгения \\
		Факультет химии, 1 курс}
\end{flushright}  	

\section*{Вопрос №1}
	\begin{enumerate}[label=\arabic*)]
		\item \textbf{Основные задачи химической термодинамики.} \\
		Термодинамика изучает количественные соотношения между
		теплотой, работой и различными формами энергии, в том числе
		и химической. \\
		Химическая термодинамика изучает превращение энергии
		химических реакций в теплоту и работу. \\
		Основные задачи:
		\begin{itemize}
			\item Определение условий реализации химических процессов. Вычисление тепловых эффектор химических реакций.
			\item Поиск пределов устойчивости исследуемых веществ при заданных условиях.
			\item Избрание оптимального режима проведения процесса.	
		\end{itemize}
	
		\item \textbf{Термодинамические параметры.} \\
		\begin{itemize}
			\item Энтальпия
			\item Энтропия
			\item Энергия Гиббса
			\item Теплоемкость
		\end{itemize}
	
		\item \textbf{Классификация систем.} \\
		Под термодинамической системой (ТС) понимается некоторая часть пространства со всеми включенными в нее компонентами. Всякая ТС должна быть ограничена реальной или вооброжаемой границей - поверхностью раздела. Через поверхность может осуществляться обмен веществом или энергией.
		\begin{itemize}
			\item Открытая система -- ТС, в которой разрешены все типы обмена.
			\item Замкнутая система -- ТС, в которой разрешен только обмен энергией.
			\item Адиабатическая система -- ТС, в которой разрешен только обмен веществом.
			\item Изолированная система -- ТС, в которой запрещены любые обмены.
		\end{itemize}
	
		\item \textbf{Экстенсивные и интенсивные термодинамические параметры.} \\
		Для полного описания системы мы выбираем минимальный набор параметров - независимых переменных (который зависит от условий эксперимента). Назовем их параметрами состояния.
		\begin{itemize}
			\item Экстенсивные параметры состояния (ЭПС) - параметры, определяемые количеством вещества в системе: $ V, m, l, S, q  $. ЭПС аддитивны.
			\item Интенсивные параметры состояния (ИПС) - параметры, которые не зависят от количества вещества и могут быть измерены лишь опосредовано через ЭПС. Примеры: $p, T $
		\end{itemize}
		Любые виды работы ($A$) могут быть охарактеризованы так: (где Y - ИПС, а Х - ЭПС):
		$$ dA = Y dX  $$ 
		Например:
		$$ dA = P dV  $$
		\item \textbf{Внутренняя энергия.} \\
		Каждое вещество характеризуется потенциальным запасом энергии. Она включает в себя все виды энергии движения и взаимодействия частиц. Эта величина - внутренняя энергия ($U$). Она зависит только от термодинамических параметров ТС и не зависит от путей достижения конкретного состояния ТС, а значит $U$ - функция состояния. \\
		Мы никогда не работаем с абсолютными значениями $U$, а лишь с ее изменениями $\Delta{U}$.
		
	\end{enumerate}	
\section*{Вопрос №2}
	\begin{enumerate}[label=\arabic*)]
		\item \textbf{Первый закон термодинамики.} \\
		Приведу две формулировки:
		\begin{itemize}
			\item В ходе любого процесса изменение внутренней энергии ТС равно разности между количеством сообщенной ей теплоты ($Q$) и совершенной ею работой ($A$):
			$$ (U_2 - U_1) = \Delta{U} = Q - A $$
			\item Сообщенная ТС теплота ($Q$) расходуется на изменение внутренней энергии ТС и совершение работы ($A$):
			$$ Q = \Delta{U} + A $$
		\end{itemize}
		\item \textbf{Функция энтальпии.} \\	
		Энтальпия ($H$) - функция состояния:
		$$ H = U + pV $$
		Использование энтальпии целесообразно, если ТС совершает  работу по сжатию/расширению, т.к. при $p = const$:
		$$ A = p \Delta{V} = \Delta{(pV)}  $$ 
		Добавим изменение внутренней энергии:
		$$Q = \Delta{U} + A = \Delta{U} + \Delta{(pV)} = \Delta{U + pV} = \Delta{H} $$
		\underline{Энтальпия - мера теплоты процесса, происходящего при постоянном давлении.} 
		
		\item \textbf{Тепловой эффект химической реакции при постоянном давлении/объеме/температуре.} \\
		\begin{itemize}
			\item При $p = const$: $Q = \Delta{H}$
			\item При $V = const$: $Q = \Delta{U}$, т.к. ТС не совершает работы.
			\item При $T = const$: $Q = A$, т.к. $U$ зависит от $T$ и при $T=const$ : $\Delta{U} = 0$.
		\end{itemize}
	
		\item \textbf{Термохимические уравнения.}\\
		Термохимическое уравнение - уравнение с указанием теплового эффекта. (!) Тепловой эффект отнесен к 1 молю вещества.\\
		В термохимических уравнениях необходимо указывать агрегатные состояния исходных веществ и продуктов реакции. \\
		Существуют экзо- и энотермические реакции:
		\begin{itemize}
			\item Экзотермические реакции происходят с выделением тепла ($\Delta{U}, \Delta{H} < 0$)
			\item Эндотермические реакции происходят с поглощением тепла ($\Delta{U}, \Delta{H} > 0$)
		\end{itemize}
		\item \textbf{Теплоты образования и сгорания. Стандартные теплоты и стандартные состояния.}
		\begin{itemize}
			\item Теплота образования -- тепловой эффект реакции образования 1 моля вещества из простых веществ.
			\item Теплота сгорания -- тепловой эффект реакции сгорания одного моля вещества в кислороде до образования оксидов в высшей степени окисления. Теплота сгорания негорючих веществ принимается равной нулю.
			\item Стандартная теплота образования -- тепловой эффект реакции образования одного моля вещества из простых веществ, его составляющих, находящихся в \underline{устойчивых стандартных состояниях.}
			\item Cтандартные состояния -- условно принятые состояния индивидуальных веществ и компонентов растворов при оценке термодинамических величин. \\ 
			Например, для стандартных условий стандартное состояние углерода -- графит, т.к. для стандартных $p$ и $T$ это равновесная модификация углерода.
			Стандартные условия:
			\begin{itemize}
				\item $p = 10^5$ Па
				\item $T = 273,15 K$
			\end{itemize}
		\end{itemize}
		\item \textbf{Энергия разрыва химической связи.}\\
		Энергия химической связи -- мольный прирост энергии вещества при разрушении одной связи определенного типа в каждой молекуле.			
	\end{enumerate}

\section*{Вопрос №3}	
	\begin{enumerate}[label=\arabic*)]
		\item \textbf{Расчет тепловых эффектов реакций по теплотам образования, сгорания и разрыва химических связей.}
		\begin{itemize}
			\item По теплотам образования: 
			$$\Delta{H_{p}^{o}} = \sum \Delta{H_{f}^o} \text{(продукты)} - \sum \Delta{H_{f}^o} \text{(реагенты)}  $$
			\item По теплотам сгорания: \\
			Аналогично первому методу, только необходимо брать энтальпию сгорания.
			\item По энергии химических связей:
			Зная состав и химическую формулу (со всеми связями между атомами вещества) можно оценить, какова его энергия формирования. Точные значения энергий конкретных связей -- справочная информация, но известно, что по силе взаимодействия:
			$$ -  <  =  <  \equiv $$
		\end{itemize}
		\item \textbf{Закон Гесса и термохимия.} \\
		Закон Гесса: \\
		Тепловой эффект химического процесса зависит только от природы и состояний исходных веществ и продуктов, но не от пути его осуществления, в том числе от выбора системы реакций и состояния промежуточных продуктов. \\
		Пример расчетов: \\
		$$C\text{(графит)} + \frac{1}{2}O_2\text{(г)} = CO\text{(г)}: \Delta{H_1}$$  
		$$C\text{(графит)} + \frac{1}{2}O_2\text{(г)} = CO_2\text{(г)}: \Delta{H_2} = -393,5 \text{кДж/моль}$$  			
		$$CO\text{(г)} + \frac{1}{2}O_2\text{(г)} = CO_2\text{(г)}: \Delta{H_3} = -110,5 \text{кДж/моль} $$  
		Отсюда:
		$$\Delta{H_1} = \Delta{H_2} - \Delta{H_3} $$
		\item \textbf{Теплоемкость.} \\		
		Истинная теплоемкость:
		$$C_T = \dfrac{dQ}{dT} $$
		Т.е. отношение теплоты ($dQ$), которая требуется, чтобы нагреть ТС на $dT$ к изменению температуры ($dT$). Тогда чтобы нагреть ТС от температуры $T_1$ до $T_2$:
		$$ Q = \int\limits_{T_1}^{T_2} C_T dT $$
		\item \textbf{Теплоемкость идеального газа.} \\
		Следует различать теплоемкость при постоянном давлении ($C_p$) и теплоемкость при постоянном объеме ($C_V$). Т.к. при $p=const$ теплота также расходуется на работу расширения. Поэтому:
		$$C_V = \dfrac{\frac{dU}{dT}}{n} $$
		$$C_p = \dfrac{\frac{dH}{dT}}{n} $$
		где $n$ - количество вещества. Для твердых и жидких веществ $C_p \approx C_V$. Для 1 моля идеального газа:
		$$C_p = \dfrac{dH}{dT} = \dfrac{d(U+pV)}{dT} = \dfrac{d(U+RT)}{dT} = \dfrac{dU}{dT} + R = C_V + R $$
		\item \textbf{Теплоемкость одноатомного и многоатомных газов.} \\
		Из МКТ известно, что:
		$$E = \dfrac{3}{2} kT \Rightarrow \Delta{U} = \dfrac{3}{2} RT $$
		Из теплоемкости идеального газа следует, что для одноатомного газа $C_V = \frac{3}{2} R $ и $C_p = \frac{5}{2}R$. На каждую поступательную степень свободы - $\frac{1}{2} R$. Столько же приходится и на вращательные. Тогда если в молекуле газа $N$ атомов $\Rightarrow 3N$ степеней свободы:
		$$ C_V = \dfrac{3N}{2} R $$
		$$ C_p = \dfrac{3N+2}{2} R $$
		Однако эти соотношения работают при больших температурах, при низких нужно учитывать вклад только вращательных степеней свободы, которых для линейных молекул 2, а для нелинейных -- 3.
		\item \textbf{Зависимость теплоемкости и энтальпии вещества от температуры.} \\
		\item \textbf{Общие понятия о фазовых переходах} \\		
		\item \textbf{Зависимость тепловых эффектов химических реакций от температуры.} \\		
		\item \textbf{Уравнение Кирхгоффа.} \\		
	\end{enumerate}
\section*{Вопрос №4}
\begin{enumerate}[label=\arabic*)]
	\item \textbf{Второй закон термодинамики} \\
	Существуют некоторые процессы, не противоречащие первому закону термодинамики, которые самопроизвольно протекать не могут.\\ \\
	Процессы, которые не могут протекать самопроизвольно -- отрицательные. Отрицательный процесс не может являться единственным результатом действия. \\
	Постулаты Клаузиса и Томсона:
	\begin{itemize}
		\item Теплота не может самопроизвольно переходить от холодного тела к горячему.
		\item Теплота более холодного из участвующих в процессе тел не может служить источником работы.
	\end{itemize}
	\item \textbf{Обратимые и необратимые процессы} \\
	\begin{itemize}
		\item Обратимый процесс -- процесс, при котором в любой фазе превращения все части рассматриваемой системы находятся в равновесии друг с другом и с внешним окружением. \\
		При обратимом процессе: 
		$$ \Delta{U}, A = const  \Rightarrow Q = const  $$
		Обратимый процесс можно осуществить единственным образом, поэтому $Q$ такого процесса  -- функция состояния.
		\item Необратимый процесс -- процесс, который нельзя провести в обратном направлении так, чтобы не произошло изменений в окружающей среде.
	\end{itemize}
	\item \textbf{Энтропия} \\
	Энтропия -- мера разупорядоченности системы.
	\begin{itemize}
		\item При обратимом процессе:
		$$\Delta{S} = S_2 - S_1 = \dfrac{Q}{T} $$ 
		\item При необратимом процессе:
		$$ dS = \dfrac{dQ}{T} + \dfrac{dI}{T} $$
		где $dI$ -- поток энергии во внешнее пространство (из-за того, что всегда $A_{HO} < A_O$, оставшаяся энергия выделяется в виде тепла), поэтому всегда:
		$$ \Delta{S} > \dfrac{Q}{T} $$
	\end{itemize}
	\item \textbf{Направление самопроизвольного процесса в изолированной системе} \\
	В изолированной системе самопроизвольно могут протекать только процессы, сопровождающиеся положительным изменением энтропии.
	\item \textbf{Статистическая природа второго закона термодинамики} \\
	\begin{itemize}
		\item Термодинамическая вероятность ($W$) -- число микросостояний, которыми мы можем реализовать данное состояние системы.
		
	\end{itemize}
	Система должна стремиться к наиболее вероятному состоянию, поэтому:
	$$ S = k \ln{W} $$
	Пример использования: \\
	При увеличении объема одного моля идеального газа в $\frac{V_2}{V_1}$ раза вероятность возрастает в $(\frac{V_2}{V_1})^N$ раз, тогда получим:
	$$ \Delta{S} = k \ln{(\frac{V_2}{V_1})^N} = k N \ln{\frac{V_2}{V_1}} = R \ln{\frac{V_2}{V_1}} $$ 
\end{enumerate}	
\section*{Вопрос №5}
\begin{enumerate}[label=\arabic*)]
	\item \textbf{Энтропия идеального кристалла} \\
	Постулат Планка: \\
	Энтропия идеального кристала при $0K$ равна $0$. \\
	В процессе охлаждения снижается амплитуда колебаний атомов в кристаллической решетке, снижается вероятность ее изменения $\Rightarrow$ понижается степень свободы $\Rightarrow$ понижается энтропия кристалла в целом. В пределе выполняется постулат Планка. 
	\item \textbf{Энтропия идеального газа} \\
	Пусть 1 моль газа нагрели от $T_1$ до $T_2$, газ расширился от $V_1$ до $V_2$. Сообщенное газу $Q$ на каждом малом участке уходит на увеличение внутренней энергии $C_V dT$ и на работу по расширению $ RT \frac{dV}{V} $, тогда суммарное изменение энтропии:
	\[
	\Delta{S} = \int\limits_{T_1}^{T_2} C_V \dfrac{dT}{T} + \int\limits_{T_1}^{T_2} R \dfrac{dV}{V} = 
	C_V \ ln{\frac{T_2}{T_1}} + R \ln{\frac{V_2}{V_1}} 
	\]
	
	\item \textbf{Изменение энтропии при постоянном объеме и постоянном давлении} \\
	\begin{itemize}
		\item При постоянном объеме работа равна 0:
		$$ 	\Delta{S} = \int\limits_{T_1}^{T_2} C_V \dfrac{dT}{T} = C_V \ ln{\frac{T_2}{T_1}} $$ 
		\item При постоянном давлении:
		$$ \Delta{Q} = \Delta{H} = C_p dT $$
		$$ \Delta{S} = \int\limits_{T_1}^{T_2} C_p \dfrac{dT}{T} = C_p \ ln{\frac{T_2}{T_1}}  $$
	\end{itemize}
	\item \textbf{Изменение энтропии в необратимых процессах} \\
	Т.к. энтропия -- функция состояния, то ее изменение будет зависеть только от начального и конечного состояния системы и одинаково для всех путей перехода между этими состояниями, включая обратимый. Поэтому в случае неравновесного процесса его следует разбить на равновесные. \\ 
	Пример: Неравновесное расширение газа против меньшего давления с нагреванием системы = равновесное расширение + нагревание при постоянном объеме.
\end{enumerate}
\section*{Вопрос №6}
\begin{enumerate}[label=\arabic*)]
	\item \textbf{Термодинамические функции} \\
	
	\item \textbf{Свободная энергия и максимальная работа} \\
	\item \textbf{Свободная энергия Гиббса и Гельмгольца} \\
	\begin{itemize}
		\item Свободная энергия Гиббса ($G$) -- та часть внутренней энергии, которую можно превратить в химическую работу \ul{при постоянных давлении и температуре.}
		\[
		G = U - TS + pV = H - TS
		\]
		\[
		\Delta{G} = \Delta{H} - T \Delta{S} - S \Delta{T}
		\]
		что при $T = const$:
		\[
		\Delta{G} = \Delta{H} - T \Delta{S} 
		\]
		\item Свободная энергия Гельмгольца ($F$) -- та часть внутренней энергии, которую можно превратить в химическую работу \ul{при постоянных объеме и температуре.} 
		\[
		F = U -TS
		\]
		\[
		\Delta{F} = \Delta{U} - T \Delta{S} - S \Delta{T}
		\]
		что при $T = const$:
		\[
		\Delta{F} = \Delta{U} - T \Delta{S} 
		\]
		
	\end{itemize}
	\item \textbf{Условия самопроизвольного протекания процесса при постоянных $V, T$ и $p, T$} \\
	\begin{itemize}
		\item $ V, T = const $ \\
		$ \Delta{F} \leqslant 0 $
		\item $ p, T = const$ \\
		$ \Delta{G} \leqslant 0 $
	\end{itemize}
	
	\item \textbf{Химический потенциал} \\
	Химический потенциал компонента системы ($\mu$) -- скорость изменения энергии Гиббса при добавлении этого компонента в систему при постоянных давлении, температуре, количествах других веществ. \\
	Для индивидуального вещества -- мольное изменение энергии Гиббса. \\
	При $p, T = const $:
	\[
	\Delta{G} = \Delta{U} - T \Delta{S} + p \Delta{V} = \mu \Delta{n}
	\]
	При $V, T = const $:
	\[
	\Delta{F} = \Delta{U} - T \Delta{S} = \mu \Delta{n}
	\]
	Продифференцировав получим:
	\[
	\mu_i = \left(\dfrac{dG}{dn_i}\right)_{p,T}
	\]
	\[
	G = U + TS + pV \Rightarrow \dfrac{dG}{dp} = V
	\]
	\ul{Пример расчета:} \\
	Для идеального газа 
	\[
	G(p_2) = G (p_1) + \int\limits_{p_1}^{p_2} V dp = G(p_1) + \int\limits_{p_1}^{p_2} nRT \dfrac{p}{dp} = G(p_1) + nRT \ln{\dfrac{p_2}{p_1}}	
	\]
	Для одного моля идеального газа:
	\[
	\mu (p_2) = \mu(p_1) + RT \ln{\dfrac{p_2}{p_1}}	
	\]
	\[
	\mu = \mu^0 + RT \ln{p}
	\]
	где $\mu^0$ -- химический потенциал газа при $p = 1$ атм.
	\item \textbf{Активность} \\
	Возьмем за меру количества вещества молярную концентрацию ($C$), тогда:
	$$ p=CRT $$
	\[
	\mu (C_2) = \mu(C_1) + RT \ln{\dfrac{C_2}{C_1}}	
	\]
	\[
	\mu = \mu^0 + RT \ln{C}
	\]
	где $\mu^0$ -- химический потенциал раствора при единичной концентрации.
	При переходе от идеальных растворов к реальным:
	\[
	\mu = \mu^0 + RT \ln{a}
	\]
	где $a = \gamma c$ -- активность. 
	\item \textbf{Термодинамические расчеты} \\
\end{enumerate}
\end{document}

\documentclass[14pt,a4paper]{scrartcl}
\renewcommand{\sfdefault}{cmr}

\usepackage[utf8]{inputenc}
\usepackage[english,russian]{babel}

\usepackage{indentfirst}
\usepackage{graphicx}
\usepackage{misccorr}
\usepackage{amsmath}
\usepackage{amssymb}
\usepackage{amsfonts}
\usepackage{icomma}
\usepackage{alltt}
\usepackage{enumitem}
\usepackage{soul}
\usepackage{soulutf8}
\usepackage{graphicx}
\usepackage{float}
\graphicspath{{pictures/}}
\DeclareGraphicsExtensions{.pdf,.png,.jpg}
\renewcommand{\AA}{\ensuremath{\mathring{A}}}

%\usepackage{titlesec}

%\titleformat
%	{\section}
%	[hang]
%	{\normalfont\bfseries}
%	{ \thesection.}{ }{}
%\titlespacing
%	{\section}
%	{\parindent}
%	{4ex}
%	{0pt}
	
%\titleformat
%	[hang]
%	{\normalfont\bfseries}
%	{ \thesection.}{ }{}
%\titlespacing
%	{\subsection}
%	{\parindent}
%	{4ex}
%	{0pt}
	
%\titleformat
%	{\subsubsection}
%	[hang]
%	{\normalfont\bfseries}
%	{ \thesection.}{ }{}
%\titlespacing
%	{\subsubsection}
%	{\parindent}
%	{4ex}
%	{0pt}

\begin{document}
	\begin{flushright}
	\textbf{Билеты по физической химии\\
		Залецкая Евгения \\
		Факультет химии, 1 курс}
\end{flushright}  	
\section*{Вопрос №1}
	
	\subsection*{Основные задачи химической термодинамики.} 
	Термодинамика изучает количественные соотношения между
	теплотой, работой и различными формами энергии, в том числе
	и химической. \\
	Химическая термодинамика изучает превращение энергии
	химических реакций в теплоту и работу. \\
	Основные задачи:
	\begin{itemize}
		\item Определение условий реализации химических процессов. Вычисление тепловых эффектор химических реакций.
		\item Поиск пределов устойчивости исследуемых веществ при заданных условиях.
		\item Избрание оптимального режима проведения процесса.	
	\end{itemize}
	
	\subsection*{Термодинамические параметры.} 
	\begin{itemize}
		\item Энтальпия
		\item Энтропия
		\item Энергия Гиббса
		\item Теплоемкость
	\end{itemize}
	
	\subsection*{Классификация систем.} 
	Под термодинамической системой (ТС) понимается некоторая часть пространства со всеми включенными в нее компонентами. Всякая ТС должна быть ограничена реальной или вооброжаемой границей - поверхностью раздела. Через поверхность может осуществляться обмен веществом или энергией.
	\begin{itemize}
		\item Открытая система -- ТС, в которой разрешены все типы обмена.
		\item Замкнутая система -- ТС, в которой разрешен только обмен энергией.
		\item Адиабатическая система -- ТС, в которой разрешен только обмен веществом.
		\item Изолированная система -- ТС, в которой запрещены любые обмены.
	\end{itemize}
	
	\subsection*{Экстенсивные и интенсивные термодинамические параметры.} 
	Для полного описания системы мы выбираем минимальный набор параметров - независимых переменных (который зависит от условий эксперимента). Назовем их параметрами состояния.
	\begin{itemize}
		\item Экстенсивные параметры состояния (ЭПС) - параметры, определяемые количеством вещества в системе: $ V, m, l, S, q  $. ЭПС аддитивны.
		\item Интенсивные параметры состояния (ИПС) - параметры, которые не зависят от количества вещества и могут быть измерены лишь опосредовано через ЭПС. Примеры: $p, T $
	\end{itemize}
	Любые виды работы ($A$) могут быть охарактеризованы так: (где Y - ИПС, а Х - ЭПС):
	$$ dA = Y dX  $$ 
	Например:
	$$ dA = P dV  $$
	\subsection*{Внутренняя энергия.} 
	Каждое вещество характеризуется потенциальным запасом энергии. Она включает в себя все виды энергии движения и взаимодействия частиц. Эта величина - внутренняя энергия ($U$). Она зависит только от термодинамических параметров ТС и не зависит от путей достижения конкретного состояния ТС, а значит $U$ - функция состояния. \\
	Мы никогда не работаем с абсолютными значениями $U$, а лишь с ее изменениями $\Delta{U}$.
		


	\section*{Вопрос №2}
	\subsection*{Первый закон термодинамики.} 
	Приведу две формулировки:
	\begin{itemize}
		\item В ходе любого процесса изменение внутренней энергии ТС равно разности между количеством сообщенной ей теплоты ($Q$) и совершенной ею работой ($A$):
		$$ (U_2 - U_1) = \Delta{U} = Q - A $$
		\item Сообщенная ТС теплота ($Q$) расходуется на изменение внутренней энергии ТС и совершение работы ($A$):
		$$ Q = \Delta{U} + A $$
	\end{itemize}
	\subsection*{Функция энтальпии.} 	
	Энтальпия ($H$) - функция состояния:
	$$ H = U + pV $$
	Использование энтальпии целесообразно, если ТС совершает  работу по сжатию/расширению, т.к. при $p = const$:
	$$ A = p \Delta{V} = \Delta{(pV)}  $$ 
	Добавим изменение внутренней энергии:
	$$Q = \Delta{U} + A = \Delta{U} + \Delta{(pV)} = \Delta{U + pV} = \Delta{H} $$
	\underline{Энтальпия - мера теплоты процесса, происходящего при постоянном давлении.} 
	
	\subsection*{Тепловой эффект химической реакции при постоянном давлении/объеме/температуре.} 
	\begin{itemize}
		\item При $p = const$: $Q = \Delta{H}$
		\item При $V = const$: $Q = \Delta{U}$, т.к. ТС не совершает работы.
		\item При $T = const$: $Q = A$, т.к. $U$ зависит от $T$ и при $T=const$ : $\Delta{U} = 0$.
	\end{itemize}
	
	\subsection*{Термохимические уравнения.}
	Термохимическое уравнение - уравнение с указанием теплового эффекта. (!) Тепловой эффект отнесен к 1 молю вещества.\\
	В термохимических уравнениях необходимо указывать агрегатные состояния исходных веществ и продуктов реакции. \\
	Существуют экзо- и энотермические реакции:
	\begin{itemize}
		\item Экзотермические реакции происходят с выделением тепла ($\Delta{U}, \Delta{H} < 0$)
		\item Эндотермические реакции происходят с поглощением тепла ($\Delta{U}, \Delta{H} > 0$)
	\end{itemize}
	\subsection*{Теплоты образования и сгорания. Стандартные теплоты и стандартные состояния.}
	\begin{itemize}
		\item Теплота образования -- тепловой эффект реакции образования 1 моля вещества из простых веществ.
		\item Теплота сгорания -- тепловой эффект реакции сгорания одного моля вещества в кислороде до образования оксидов в высшей степени окисления. Теплота сгорания негорючих веществ принимается равной нулю.
		\item Стандартная теплота образования -- тепловой эффект реакции образования одного моля вещества из простых веществ, его составляющих, находящихся в \underline{устойчивых стандартных состояниях.}
		\item Cтандартные состояния -- условно принятые состояния индивидуальных веществ и компонентов растворов при оценке термодинамических величин.  
		Например, для стандартных условий стандартное состояние углерода -- графит, т.к. для стандартных $p$ и $T$ это равновесная модификация углерода.
		Стандартные условия:
		\begin{itemize}
			\item $p = 10^5$ Па
			\item $T = 273,15 K$
		\end{itemize}
	\end{itemize}
	\subsection*{Энергия разрыва химической связи.}
	Энергия химической связи -- мольный прирост энергии вещества при разрушении одной связи определенного типа в каждой молекуле.			


\section*{Вопрос №3}	

		\subsection*{Расчет тепловых эффектов реакций по теплотам образования, сгорания и разрыва химических связей.}
		\begin{itemize}
			\item По теплотам образования: 
			$$\Delta{H_{p}^{o}} = \sum \Delta{H_{f}^o} \text{(продукты)} - \sum \Delta{H_{f}^o} \text{(реагенты)}  $$
			\item По теплотам сгорания: \\
			Аналогично первому методу, только необходимо брать энтальпию сгорания.
			\item По энергии химических связей:
			Зная состав и химическую формулу (со всеми связями между атомами вещества) можно оценить, какова его энергия формирования. Точные значения энергий конкретных связей -- справочная информация, но известно, что по силе взаимодействия:
			$$ -  <  =  <  \equiv $$
		\end{itemize}
		\subsection*{Закон Гесса и термохимия.} 
		Закон Гесса: \\
		Тепловой эффект химического процесса зависит только от природы и состояний исходных веществ и продуктов, но не от пути его осуществления, в том числе от выбора системы реакций и состояния промежуточных продуктов. \\
		Пример расчетов: \\
		$$C\text{(графит)} + \frac{1}{2}O_2\text{(г)} = CO\text{(г)}: \Delta{H_1}$$  
		$$C\text{(графит)} + \frac{1}{2}O_2\text{(г)} = CO_2\text{(г)}: \Delta{H_2} = -393,5 \text{кДж/моль}$$  			
		$$CO\text{(г)} + \frac{1}{2}O_2\text{(г)} = CO_2\text{(г)}: \Delta{H_3} = -110,5 \text{кДж/моль} $$  
		Отсюда:
		$$\Delta{H_1} = \Delta{H_2} - \Delta{H_3} $$
		\subsection*{Теплоемкость.} 		
		Истинная теплоемкость:
		$$C_T = \dfrac{dQ}{dT} $$
		Т.е. отношение теплоты ($dQ$), которая требуется, чтобы нагреть ТС на $dT$ к изменению температуры ($dT$). Тогда чтобы нагреть ТС от температуры $T_1$ до $T_2$:
		$$ Q = \int\limits_{T_1}^{T_2} C_T dT $$
		\subsection*{Теплоемкость идеального газа.} 
		Следует различать теплоемкость при постоянном давлении ($C_p$) и теплоемкость при постоянном объеме ($C_V$). Т.к. при $p=const$ теплота также расходуется на работу расширения. Поэтому:
		$$C_V = \dfrac{\frac{dU}{dT}}{n} $$
		$$C_p = \dfrac{\frac{dH}{dT}}{n} $$
		где $n$ - количество вещества. Для твердых и жидких веществ $C_p \approx C_V$. Для 1 моля идеального газа:
		$$C_p = \dfrac{dH}{dT} = \dfrac{d(U+pV)}{dT} = \dfrac{d(U+RT)}{dT} = \dfrac{dU}{dT} + R = C_V + R $$
		\subsection*{Теплоемкость одноатомного и многоатомных газов.} 
		Из МКТ известно, что:
		$$E = \dfrac{3}{2} kT \Rightarrow \Delta{U} = \dfrac{3}{2} RT $$
		Из теплоемкости идеального газа следует, что для одноатомного газа $C_V = \frac{3}{2} R $ и $C_p = \frac{5}{2}R$. На каждую поступательную степень свободы - $\frac{1}{2} R$. Столько же приходится и на вращательные. Тогда если в молекуле газа $N$ атомов $\Rightarrow 3N$ степеней свободы:
		$$ C_V = \dfrac{3N}{2} R $$
		$$ C_p = \dfrac{3N+2}{2} R $$
		Однако эти соотношения работают при больших температурах, при низких нужно учитывать вклад только вращательных степеней свободы, которых для линейных молекул 2, а для нелинейных -- 3.
		\subsection*{Зависимость теплоемкости и энтальпии вещества от температуры.} 
		\subsection*{Общие понятия о фазовых переходах} 		
		\subsection*{Зависимость тепловых эффектов химических реакций от температуры.} 		
		\subsection*{Уравнение Кирхгоффа.} 		


	\section*{Вопрос №4}
	
	\subsection*{Второй закон термодинамики} 
	Существуют некоторые процессы, не противоречащие первому закону термодинамики, которые самопроизвольно протекать не могут.\\ \\
	Процессы, которые не могут протекать самопроизвольно -- отрицательные. Отрицательный процесс не может являться единственным результатом действия. \\
	Постулаты Клаузиса и Томсона:
	\begin{itemize}
		\item Теплота не может самопроизвольно переходить от холодного тела к горячему.
		\item Теплота более холодного из участвующих в процессе тел не может служить источником работы.
	\end{itemize}
	\subsection*{Обратимые и необратимые процессы} 
	\begin{itemize}
		\item Обратимый процесс -- процесс, при котором в любой фазе превращения все части рассматриваемой системы находятся в равновесии друг с другом и с внешним окружением. \\
		При обратимом процессе: 
		$$ \Delta{U}, A = const  \Rightarrow Q = const  $$
		Обратимый процесс можно осуществить единственным образом, поэтому $Q$ такого процесса  -- функция состояния.
		\item Необратимый процесс -- процесс, который нельзя провести в обратном направлении так, чтобы не произошло изменений в окружающей среде.
	\end{itemize}
	\subsection*{Энтропия} 
	Энтропия -- мера разупорядоченности системы.
	\begin{itemize}
		\item При обратимом процессе:
		$$\Delta{S} = S_2 - S_1 = \dfrac{Q}{T} $$ 
		\item При необратимом процессе:
		$$ dS = \dfrac{dQ}{T} + \dfrac{dI}{T} $$
		где $dI$ -- поток энергии во внешнее пространство (из-за того, что всегда $A_{HO} < A_O$, оставшаяся энергия выделяется в виде тепла), поэтому всегда:
		$$ \Delta{S} > \dfrac{Q}{T} $$
	\end{itemize}
	\subsection*{Направление самопроизвольного процесса в изолированной системе} 
	В изолированной системе самопроизвольно могут протекать только процессы, сопровождающиеся положительным изменением энтропии.
	\subsection*{Статистическая природа второго закона термодинамики} 
	\begin{itemize}
		\item Термодинамическая вероятность ($W$) -- число микросостояний, которыми мы можем реализовать данное состояние системы.
		
	\end{itemize}
	Система должна стремиться к наиболее вероятному состоянию, поэтому:
	$$ S = k \ln{W} $$
	Пример использования: \\
	При увеличении объема одного моля идеального газа в $\frac{V_2}{V_1}$ раза вероятность возрастает в $(\frac{V_2}{V_1})^N$ раз, тогда получим:
	$$ \Delta{S} = k \ln{(\frac{V_2}{V_1})^N} = k N \ln{\frac{V_2}{V_1}} = R \ln{\frac{V_2}{V_1}} $$
	
\section*{Вопрос №5}
	
	\subsection*{Энтропия идеального кристалла} 
	Постулат Планка: \\
	Энтропия идеального кристала при $0K$ равна $0$. \\
	В процессе охлаждения снижается амплитуда колебаний атомов в кристаллической решетке, снижается вероятность ее изменения $\Rightarrow$ понижается степень свободы $\Rightarrow$ понижается энтропия кристалла в целом. В пределе выполняется постулат Планка. 
	\subsection*{Энтропия идеального газа} 
	Пусть 1 моль газа нагрели от $T_1$ до $T_2$, газ расширился от $V_1$ до $V_2$. Сообщенное газу $Q$ на каждом малом участке уходит на увеличение внутренней энергии $C_V dT$ и на работу по расширению $ RT \frac{dV}{V} $, тогда суммарное изменение энтропии:
	\[
	\Delta{S} = \int\limits_{T_1}^{T_2} C_V \dfrac{dT}{T} + \int\limits_{T_1}^{T_2} R \dfrac{dV}{V} = 
	C_V \ ln{\frac{T_2}{T_1}} + R \ln{\frac{V_2}{V_1}} 
	\]
	
	\subsection*{Изменение энтропии при постоянном объеме и постоянном давлении} 
	\begin{itemize}
		\item При постоянном объеме работа равна 0:
		$$ 	\Delta{S} = \int\limits_{T_1}^{T_2} C_V \dfrac{dT}{T} = C_V \ ln{\frac{T_2}{T_1}} $$ 
		\item При постоянном давлении:
		$$ \Delta{Q} = \Delta{H} = C_p dT $$
		$$ \Delta{S} = \int\limits_{T_1}^{T_2} C_p \dfrac{dT}{T} = C_p \ ln{\frac{T_2}{T_1}}  $$
	\end{itemize}
	\subsection*{Изменение энтропии в необратимых процессах} 
	Т.к. энтропия -- функция состояния, то ее изменение будет зависеть только от начального и конечного состояния системы и одинаково для всех путей перехода между этими состояниями, включая обратимый. Поэтому в случае неравновесного процесса его следует разбить на равновесные. \\ 
	Пример: Неравновесное расширение газа против меньшего давления с нагреванием системы = равновесное расширение + нагревание при постоянном объеме.

\section*{Вопрос №6}

	\subsection*{Термодинамические функции} 
	
	\subsection*{Свободная энергия Гиббса и Гельмгольца}\
	\begin{itemize}
		\item Свободная энергия Гиббса ($G$) -- та часть внутренней энергии, которую можно превратить в химическую работу \ul{при постоянных давлении и температуре.}
		\[
		G = U - TS + pV = H - TS
		\]
		\[
		\Delta{G} = \Delta{H} - T \Delta{S} - S \Delta{T}
		\]
		что при $T = const$:
		\[
		\Delta{G} = \Delta{H} - T \Delta{S} 
		\]
		\item Свободная энергия Гельмгольца ($F$) -- та часть внутренней энергии, которую можно превратить в химическую работу \ul{при постоянных объеме и температуре.} 
		\[
		F = U -TS
		\]
		\[
		\Delta{F} = \Delta{U} - T \Delta{S} - S \Delta{T}
		\]
		что при $T = const$:
		\[
		\Delta{F} = \Delta{U} - T \Delta{S} 
		\]
		
	\end{itemize}
	\subsection*{Условия самопроизвольного протекания процесса при постоянных $V, T$ и $p, T$} 
	\begin{itemize}
		\item $ V, T = const $ \\
		$ \Delta{F} \leqslant 0 $
		\item $ p, T = const$ \\
		$ \Delta{G} \leqslant 0 $
	\end{itemize}
	
	\subsection*{Химический потенциал} 
	Химический потенциал компонента системы ($\mu$) -- скорость изменения энергии Гиббса при добавлении этого компонента в систему при постоянных давлении, температуре, количествах других веществ. \\
	Для индивидуального вещества -- мольное изменение энергии Гиббса. \\
	При $p, T = const $:
	\[
	\Delta{G} = \Delta{U} - T \Delta{S} + p \Delta{V} = \mu \Delta{n}
	\]
	При $V, T = const $:
	\[
	\Delta{F} = \Delta{U} - T \Delta{S} = \mu \Delta{n}
	\]
	Продифференцировав получим:
	\[
	\mu_i = \left(\dfrac{dG}{dn_i}\right)_{p,T}
	\]
	\[
	G = U + TS + pV \Rightarrow \dfrac{dG}{dp} = V
	\]
	\ul{Пример расчета:} \\
	Для идеального газа 
	\[
	G(p_2) = G (p_1) + \int\limits_{p_1}^{p_2} V dp = G(p_1) + \int\limits_{p_1}^{p_2} nRT \dfrac{p}{dp} = G(p_1) + nRT \ln{\dfrac{p_2}{p_1}}	
	\]
	Для одного моля идеального газа:
	\[
	\mu (p_2) = \mu(p_1) + RT \ln{\dfrac{p_2}{p_1}}	
	\]
	\[
	\mu = \mu^0 + RT \ln{p}
	\]
	где $\mu^0$ -- химический потенциал газа при $p = 1$ атм.
	\subsection*{Активность} 
	Возьмем за меру количества вещества молярную концентрацию ($C$), тогда:
	$$ p=CRT $$
	\[
	\mu (C_2) = \mu(C_1) + RT \ln{\dfrac{C_2}{C_1}}	
	\]
	\[
	\mu = \mu^0 + RT \ln{C}
	\]
	где $\mu^0$ -- химический потенциал раствора при единичной концентрации.
	При переходе от идеальных растворов к реальным:
	\[
	\mu = \mu^0 + RT \ln{a}
	\]
	где $a = \gamma c$ -- активность. 
	\subsection*{Термодинамические расчеты} 
\section*{Вопрос №7}
\subsection*{Обратимость химических реакций}
Обратимые химические реакции -- реакции, которые одновременно идут в двух взаимно противоположных направлениях. \\
\ul{Не путать с обратимыми процессами:} обратимая реакция не обязательно должна протекать в равновесных условиях.
\subsection*{Химическое равновесие}
Химическое равновесие -- состояние системы, при котором скорость прямой реакции равна скорости обратной. \\
При состоянии химического равновесия реакция не прекращается, а обе реакции идут с равными скоростями. \\
Математическое выражение условия равновесия:
\[
\Delta{H} = T \Delta{S} \Rightarrow \Delta{G} = 0
\]
\subsection*{Условия химического равновесия в гомо- и гетерогенных системах}
Для гомогенных процессов:
\begin{itemize}
	\item Равновесие в реакция, протекающих в газовой фазе: \\
	Пусть есть реакция:
	$$ aA + bB = cC +  dD $$
	Тогда условие равновесия:
	$$ \Delta{G} = cG_c + dG_D - aG_A - bG_b $$
	Выразим химические потенциалы через стандартые с учетом давления (т.к. речь про газы):
	$$ \mu_i = \mu^o_i + RT \ln{p} $$
	Тогда:
	$$ \Delta{G} = c\mu_C + d\mu_D - a\mu_A - b\mu_B = $$
	$$ =c\mu^o_C + d\mu^o_D - a\mu^o_A - b^o\mu_B + RT \ln{(c\ln{p_C} + d\ln{p_D} - a\ln{p_A} - b\ln{p_B})} = $$
	$$ = \Delta{G^o} + RT \ln{\left[ \dfrac{(p_C)^c(p_D)^d}{(p_A)^a(p_B)^b}\right]} $$
	В условиях равновесия величина в квадратных скобках -- константа равновесия. Поскольку при равновесии $\Delta{G} = 0$:
	$$ \ln{\left[ \dfrac{(p_C)^c(p_D)^d}{(p_A)^a(p_B)^b}\right]} = \ln{K_p} = - \dfrac{\Delta{G^o}}{RT}  $$
	\item Равновесие в реакциях, протекающих в растворах:\\
	Расчет аналогичен газам, только вместо давления следует рассматривать зависимость потенциала от концентрации или активности.
\end{itemize}
Для гетерогенных процессов: \\
Пусть есть процесс разложения карбоната кальция:
$$ CaCO_3 = CaO + CO_2 $$
Для равновесия при атмосферном давлении:
$$ \Delta{G} = \mu_{CO_2} + \mu_{CaO} - \mu_{CaCO_3} = \mu_{CO_2}^o + RT \ln{p_{CO_2}} + \mu_{CaO}^o - \mu_{CaCO_3}^o = $$
$$ = \Delta{G^o} + \ln{p_{CO_2}} $$
Т.е. константой равновесия данного процесса будет парциальное давление $CO_2$. \\
В общем случае: если гетерогенный процесс происходит с участием газов, нужно учитывать лишь  их парциальное давление. Если процесс в растворах с выделением/расстворением осадка -- аналогично расчетам гомогенных реакций в расстворах.
\subsection*{Глубина протекания процессов}

\subsection*{Степень превращения}
\subsection*{Константа равновесия}
Состояние химического равновесия количественно характеризуется константой равновесия, представляющей собой отношение констант прямой ($K_1$) и обратной ($K_2$) реакций. Таким образом, для реакции
$$ aA + bB = cC +  dD $$
Константа равновесия:
$$ K_p = \frac{K_1}{K_2} = \dfrac{\left[ C\right]^c \left[ D\right]^d }{\left[ A\right]^a \left[ B\right]^b} $$
\subsection*{Факторы, влияющие на величину константы равновесия}
Константа равновесия не зависит от давления и концентраций, но зависит от температуры.
\subsection*{Смещение положения равновесия, принцип подвижного равновесия Ле Шателье - Брауна}
Если на систему, находящуюся в состоянии подвижного равновесия, оказывается внешнее воздействие, то положение равновесия смещается в сторону, противодействующую этому воздействию. \\
Пример: \\
\begin{itemize}
	\item $ \Delta{H} > 0, \Delta{n} = 0 $ 
	$$ H_2 + I_2 = 2HI $$
	При увеличении температуры равновесие смещается в сторону продуктов. \\
	Давление никак не влияет на равновесие, т.к. одинаковое количество молекул газа и слева, и справа.
	\item $ \Delta{H} < 0, \Delta{n} < 0 $
	$$ 3H_2 + N_2 = 2 NH_3  $$
	При увеличении температуры равновесие смещается в сторону реагентов. \\
	При увеличение давления -- в сторону продуктов, т.к. справа 2 молекулы газа, а слева -- 4.
	\item $ \Delta{H} > 0, \Delta{n} > 0 $
	$$ N_2O_4 = 2NO_2 $$
	При увеличении температуры равновесие смещается в сторону продуктов. \\
	При увеличении давления -- в сторону реагентов.  	
\end{itemize}

\subsection*{Стандартная свободная энергия}
Это разность между свободной энергией исходных веществ и свободной энергией продуктов реакции.

\section*{Вопрос № 8}
Немножко определений:
\begin{itemize}
	\item Гомогенная система -- система, внутри которой нет поверхности раздела.
	\item Гетерогенная система -- система, в которой можно выделить части, различающиеся по строению, свойствам и отделенные друг от друга границами раздела.
	\item Фаза ($f$) -- гомогенная часть гетерогенной системы, отделенная от других поверхностью раздела и отличающаяся от них по структуре и свойствам.
	\item Конденсированные системы -- система, в которой газовая фаза не оказывает существенного влияния на фазовые и химические равновесия.
	\item Число независимых компонентов ($k$) -- минимальное число компонентов, из которых можно организовать данную систему. Оно равно общему числу компонентов за вычетом количества связывающих их уравнений.
\end{itemize}
\subsection*{Фазовые равновесия}
\subsection*{Теплоты кипения и плавления}
\subsection*{Термический анализ}
\subsection*{Правило фаз Гиббса}
Число степеней свободы равновесной термодинамической системы превышет разность между числом независимых компонентов и количеством фаз на два.
$$ f+c = k+2 $$
Двойка здесь -- влияние температуры и давления. В случае рассмотрения конденсированных систем, правило фаз будет выглядеть так:
$$ f+c = k+1 $$
Т.к. давление не оказывает существенного воздействия. \\
Если нам важно учитывать еще каких-нибудь $n$ параметров (например, электромагнитное поле):
$$ f+c = k+2 + n $$

\subsection*{Степень свободы, вариантность системы}
Число степеней свободы (вариантность системы) -- максимальное число переменных, которые можно менять независимо от других, не меняя при этом состояния системы.
\subsection*{Фазовые диаграммы однокомпонентных систем}
Фазовая диаграмма -- схема, отражающая области термодинамической стабильности существующих в системе фаз и обозначающая границы раздела между ними.\\
Фазовое поле -- область фазовой диаграммы, в которой число и качественный состав фаз не меняется. \\
Для однокомпонентных систем ($f = 1, c = 2$):
$$ f+c = 3 $$
\subsection*{Фазовые равновесия на $pT$ - диаграмме}
\subsection*{$pT$ - диаграмма воды}
\subsection*{Фазовые поля}
Фазовое поле -- область фазовой диаграммы, в которой число и качественный состав фаз не меняется.
\subsection*{Тройная точка}
Тройная точка -- точка на фазовой диаграмме, в которой сосуществуют три фазы. Для однокомпонентной системы число степеней свободы в тройной точке (из правила фаз Гиббса) равно 0. Равновесие в этой точке -- инвариантное.
\subsection*{Метастабильные состояния}
\subsection*{Фазовые переходы первого рода}
\subsection*{$pT$ - диаграмма серы}
	
\section*{Вопрос №10}
\subsection*{Насыщенный раствор и растворимость}
\begin{itemize}
	\item Насыщенный раствор -- раствор, в котором при данной температуре данное вещество уже больше не растворяется.
	\item Растворимость -- способность вещества образовыввать с другими веществами однородные системы – растворы, в которых вещество находится в виде отдельных атомов, ионов, молекул или частиц. \\
	астворимость выражается концентрацией растворенное вещества в его насыщенном растворе либо в процентах, либо в весовых или объемных единицах, отнесенных к 100 г или 100 мл. Растворимость газов в жидкости зависит от температуры и давления.  Растворимость жидких и твердых веществ – только от температуры.
\end{itemize}

\subsection*{Диаграмма системы соль-вода}
\begin{figure}[htp]
\centering
\includegraphics[scale=1.5]{h2o-diagram.png}
\caption{Диаграмма системы соль-вода на примере сульфата натрия}
\label{}
\end{figure}


Излом на графике обусловлен тем, что растворимость $Na_2SO_4\cdot10H_2O$ является эндотермическим процессом. В то же время при 32 градусах он инконгруэнтно (с разложением) плавится. Выше этой температуры величина растворимости обусловлена равновесием раствора с безводным сульфатом натрия. Переход последнего в гидрат сопряжен со значительным выделением тепла.  Поэтому для $Na_2SO_4\cdot10H_2O$ энтальпия положительна. До тех пор, пока растворимость определяется равновесием декагидрата с раствором, она растет, а после точки перетектики – падает с ростом температуры.

\subsection*{Зависимость растворимости от температуры}

У одних солей растворимость очень сильно растет с ростом температуры, у других менее резко. \\
У кристаллических веществ может наблюдаться не только увеличение, но и понижение растворимости при нагревании. Это связано со знаком изменения энтальпии в процессе
растворения. Разрушение кристаллической структуры вещества при переходе в раствор требует затраты энергии ( эндотермический процесс). Но частицы растворенного вещества химически взаимодействуют с растворителями. Химическое взаимодействие в растворе сопровождается уменьшением энтальпии(экзотермический процесс). Суммарное изменение энтальпии процесса растворения может оказаться как положительным, так и отрицательным. \\
При протекании эндотермического процесса происходит увеличение растворимости с ростом температуры.

\subsection*{Факторы, влияющие на растворимость}
\begin{itemize}
\item Растворимость веществ во многом обуславливается силой и характером их взаимодействия с молекулами растворителя.

\item Растворимость веществ значительно повышается, если они способны образовывать с растворителем водородные и донорно-акцепторные связи.

\item Диэлектрическая проницаемость растворителя является одним из наиболее важных факторов, влияющих на растворимость.

\item Природа смешиваемых веществ (полярные с полярными, неполярные с неполярными)
\end{itemize}
\subsection*{Криогидратная точка}
Криогидратная точка -- точка, соответствующая эвтектике для системы соль-вода.



\section*{Вопрос №12}
\subsection*{Коллигативные свойства растворов}
Коллигативные свойства раствора -- такие свойства, которые зависят лишь от количества частиц растворенного вещества в единице объема. \\
К ним относят: осмотическое давление, повышение температуры кипения, понижение температуры замерзания.
\subsection*{Крио и эбулиоскопия}
\begin{itemize}
	\item Криоскопия -- метод исследования растворов, в основе которого лежит измерение понижения точки замерзания раствора по сравнению с температурой замерзания чистого растворителя. \\
	Понижение температуры замерзания пропорционально мольной доле растворенного вещества. \\
	\item Эбулиоскопия -- метод исследования растворов, основанный на измерении повышения их температуры кипения по сравнению с чистым растворителем. \\
	Введение в растворитель некоторого вещества приводит к дополнительному повышению энтропии за счет разрушения кристаллической решетки. Следствием этого является понижения химиеческого потенциала жидкости:
	$$ \mu = \mu_a^0 + RT \ln{x_a} = \mu_a^0 + RT \ln{1- x_s}   $$
	где $x_s$ - мольная доля растворенного вещества. \\
	Т.о. выравнивание потенциала жидкости с химическим потенциалом паров растворителя будет происходить при более высоких температурах, что соответсвует повышению температуры кипения с ростом концентрации растворенного вещества.
	
	
	
\end{itemize}
\subsection*{Осмос и осмотическое давление}
Осмос -- процесс односторонней диффузии растворителя через полупроницаемую перегородку от раствора с меньшей концентрацией растворенного вещества к раствору с большей концентрацией. \\
Если внести в сосуд с чистым растворителем емкость с раствором, отделенную от него полупроницаемой перегородкой, окажется, что молекулы растворителя преимущественно диффундируют в раствор до тех пор, пока уровень в сосуде с раствором не поднимется на некоторую высоту. Разница уровней столбов жидкости уравновешивает процесс диффузии растворителя.\\
Осмотическое давление -- величина давления, которую надо приложить к раствору, чтобы предотвратить диффузию в него через полупроницаемую перегородку чистого растворителя. 
\subsection*{Термодинамическое обоснование закона Вант Гоффа}
При условии равновесия в двух сосудах должно выполняться условия равенства химических потенциалов растворителя ($\mu_a^1 = \mu_a^0$). Продифференцируем это уравнение по мольной доле растворителя $x_a$ и давления $p$ в сосуде с растворителем:
$$ d\mu_a^1 = \left(\dfrac{d\mu_a^1}{dx_a} \right)_{p,T}dx_a + \left(\dfrac{d\mu_a^1}{dp} \right)_{x,T}dp = d\mu^0  $$
Поскольку давление и концентрация во втором сосуде не влияют на первый, химический потенциал чистого растворителя остается постоянным и $d\mu^0 =0 $. Кроме того:
$$ \left(\dfrac{ d\mu_a^1 }{dp}\right) = \dfrac{d^2G_a}{dx} dp = \left( \dfrac{dV_a}{dx} \right) = \underline{V}_a  $$
и
$$ \left(\dfrac{ d\mu_a^1 }{dx_a}\right) = RT \left[d\ln{\dfrac{x_a}{dx_a}}\right] $$
откуда:
$$ RT d \ln{x_a} + \underline{V}_a  dp = 0  $$
$$ dp = - \dfrac{RT}{\underline{V}_a} d \ln{x_a} $$
где $\underline{V}_a $ - мольный объем растворителя. \\
Интегрируя это уравнение от $x_a = 1, p=p_0$ до $x_a$ и $p+\pi$, где $p_0$ - внешнее давление, а $\pi$ - осмотическое давление, получаем:
$$ p_0 + \pi - p_0 = - \dfrac{RT}{\underline{V}_a} [\ln{x_a}- \ln{1}]  $$
$$ \pi = - \dfrac{RT}{\underline{V}_a} \ln{x_a} $$
При малых концентрациях растворенного вещества:
$$ \ln{x_a} = \ln{(1-x_s)} \approx -x_s, x_s \approx \dfrac{X_s}{X_a} $$
$$ X_a\underline{V}_a = V$$
Откуда:
$$ \pi = \dfrac{RT}{\underline{V}_a} \ln{x_a} = \dfrac{X_sRT}{X_a\underline{V}_a} = \left( \dfrac{X_s}{V}\right)RT = CRT $$
где $X_S$ и $X_a$ - число молей растворенного вещества и растворителя, а $C$ -- молярная концентрация. Это выражение называют уравнением Вант-Гоффа и строго выполняется только для разбавленных растворов. Для концентрированных растворов приходится использовать активность вместо концентрации.

\subsection*{Определение молекулярных масс органических соединений на основании свойств растворов}
Рассмотрим возможность такого измерения на примере эбулиоскопии. \\
Для решения поставленной задачи измеряяют температуры кипения некоторого заранее отмеренного количества растворителя после последовательного добавления к нему навесок вещества с известной молекулярной массой. \\
По графику зависимости температуры кипения от концентрации вычисляют эбулиоскопическую постоянную ($K$).
$$ \Delta{T}_{\text{кип}} = Km  $$
$m$ - моляльная концентрация. \\
Проделав аналогичную процедуру для исследуемого вещества, определяют его молекулярный вес.
\begin{figure}[H]
	\centering
	\includegraphics[scale=0.5]{cryoscopy}
	\label{}
\end{figure}
Выше приведен пример графика зависимости температуры замерзания (в случае использования криоскопии) от концентрации раствора.


\section*{Вопрос №14}
\subsection*{Растворы сильных электролитов}
Сильные электролиты, попадая в раствор, диссоциируют нацело или почти нацело (при этом понятие константы диссоциации для них неприменимо). \\
Кажущаяся степень диссоциации -- величина, характеризующая отношение активности ионов сильных электролитов к их истинной концентрации. \\
По мере концентрирования раствора ксд понижается. \\
Попадая в раствор, $NaCl$ диссоциирует полностью, т.е. нельзя сказать, что в растворе присутствует заметная концентрация $NaCl$. Однако за счет кулоновского взаимодействия образуется некое подобие упорядоченного расположения ионов, при котором катионы оказываются преимущественно в окружении анионов и наоборт (т.н. "ионная шуба"). При попадании раствора в электрическое поле равновесие нарушается. Т.к. катионы движутся к катоду, а анионы -- к аноду, "ионная шуба" располагается с противоположной к направлению движения иона стороны, поэтому движение иона тормозится.
\subsection*{Ионные пары, активность, ионная сила раствора}
\begin{itemize}
	\item Ионная пара -- пара противоположно заряженных ионов, удерживающихся вместе за счет кулоновского притяжения без образования ковалентной связи. \\
	В растворах это возможно, так как по мере увеличения концентрации межатомные расстояния уменьшаются. При этом кулоновское взаимодействие усиливается, и ионы проявляют большую склонность к ассоциации. Это выражается в том, что гидратированные катион и анион получают возможность совместного существования в растворе в течение некоторого времени. \\
	\item Активность ($a$) -- это аналог концентраций для неидеальных растворов. Основной причиной ее отклонения от концентрации являются межионные взаимодействия в растворе. \\
	Активность электролита $X_nY_m$:	
	$$ a = \sqrt[(n+m)]{a_x^n \cdot a_y^m} $$
	\item Ионная сила раствора ($I$) -- мера электрического взаимодействия между всеми ионами в растворе:
	$$ I = \dfrac{1}{2} \Sigma(Z_i^2 \cdot C_i) $$
	где  $C_i $ - концентрация $i$-того иона, а $Z_i $ - его заряд.
\end{itemize}
\subsection*{Малорастворимые соли}
Если ионная соль растворяется слабо, значит энергия кристаллической решетки в ней значительно превышает энергию гидратации.\\
Для малорастворимых веществ ПР $\ll$ 1.
\subsection*{Произведение растворимости}
Процесс растворения соединения состава $X_nY_m$ в общем случае:
$$ X_nY_m \text{(тв)} \leftrightarrows n X^{m+} \text{(р-р)} + mY^{n-} \text{(р-р)}$$
Поскольку при равновесии раствор сосуществует с твердой фазой, активность которой равна $1$, константа равновесия:
$$ \left[X^{m+} \right]^n\left[Y^{n-} \right]^m = e^{\frac{-\Delta{G}}{RT}} = 
e^{\frac{-\Delta{H}}{RT}} e^{\frac{\Delta{S}}{RT}} = \text{ПР} $$
Т.о. произведение растворимости -- константа для данного вещества и зависит только от температуры и природы растворителя. ПР растет с ростом температуры.
\subsection*{Способы понижения и повышения растворимости}
Понизить растворимость соединения можно приливанием раствора, содержащего одноимённые ион. \\
Повысить растворимость можно, связывая один из ионов, например за счёт комплексообразования. \\
Повысить растворимость можно нагреванием.




\section*{Вопрос №15}
\subsection*{Теория кислот и оснований}
\begin{itemize}
	\item \textbf{Теория Аррениуса}\\\\
	Кислотами называются вещества, которые в водном растворе диссоциируют с образованием ионов водорода, а основания - вещества, диссоциирующие с образованием ионов гидроксила.\\\\
	Позже теория была скорректирована: \\
	Протон, обладающий чрезвычайно малым радиусом порядка  $10^{-7} \AA $, не способен к самостоятельному существованию в растворе. Обладая огромной поляризующей способностью, он в любой момент времени оказывается локализованным на каком-либо электроотрицательном атоме, имеющем неподеленную электронную пару. Таким образом, в водных растворах кислот образуются ионы гидроксония ($H_3O^+ $),  в которых все три протона оказываются эквивалентными.

	\ul{Минусы:} не позволяет описать диссоциацию в апротонных растворителях
	\item \textbf{Теория Бренстеда}\\\\
	Кислоты -- вещества, отдающие протон, а основаниями - принимаюзщие его. \\\\
	В соответствии с этим определением в качестве основания можно рассматривать аммиак, присоединяющий протон по реакции:
 	$$NH_3 + H^+ \rightleftarrows NH_4^+ $$

	Соответственно ион аммония, напротив, является кислотой. В свете этой теории каждой кислоте соответствует некоторое сопряженное основание:
 	$$ A \rightleftarrows H^+ + B$$
 
 	\ul{Минусы:} не объясняет, почему водный раствор $BF_3$ является кислым
 
	\item \textbf{Теория Льюиса}\\\\
	Кислота -- вещество, которое при образовании ковалентной связи принимают пару электронов. Основание -- вещество, отдающее пару электронов при образовании такой же связи. \\\\
 	К кислотам Льюиса относятся координационно ненасыщенные молекулы или ионы, способные достраивать свою координационную сферу за счет присоединения лигандов, соединения, обладающие незавершенной восьмиэлектронной оболочкой, оксиды и галогениды с ненасыщенной координационной оболочкой ($PF_5$, $SbF_5$...) и катионы, способные образовывать комплексные соединения.
 
 	К основаниям относятся анионы, включая $OH^-$, и молекулы, способные выступать в комплексных  соединениях в качестве лигандов.
 
 	\ul{Минусы:} невозможно предсказать силу, не может описать классические кислоты.

	\item \textbf{Теория Пирсона}\\\\
	Связь между кислотой и основанием не обязательно должна являться ковалентной, а может включать и другие виды взаимодействия (в том числе и ионную связь). \\\\
	В соответствии с этой теорией к реакциям нейтрализации относится даже процесс образования кристаллической решетки ионами $Na^+$ и $Cl^-$.

	Наиболее ценным в теории Пирсона является представление о наличии жестких и мягких кислот и оснований. 

	К жестким относятся частицы с реакционноспособными центрами, обладающие мало деформируемой электронной структурой. Это могут быть катионы с большим зарядом и малым радиусом, а также анионы элементов с высокой электроотрицательностью ($F, O, N$) (или включающие эти элементы $NH_3$, $ClO_4^-, OH^-$).

	Мягкие частицы имеют электронную структуру с высокой поляризуемостью (катионы с $d$-электронной подкладкой. объемные анионы, лиганды с функциональными группами, включающими атомы с низкой электроотрицательностью и т.д.)

	Жесткие основания образуют наиболее прочные связи с жесткими кислотами, и наоборот. Так, например, такие жесткие кислоты, как ионы $B^{3+}$, $Al^{3+}$, $Ti^{4+}$, предпочтительно образуют соли с жесткими основаниями ($O^{2-}$, $F^-$). С другой стороны, такие мягкие кислоты, как $Ag^+$, $Tl^+$, напротив, более склонны к образованию солей с мягкими основаниями, например, с иодид- или сульфид-ионами.

	\ul{Минусы:} излишняя обширность. В соответствии с этой теорией к реакциям нейтрализации относится даже процесс образования кристаллической решетки ионами $Na^+$ и $Cl^-$.
\end{itemize}

\subsection*{Автопротолиз. Ионное произведение воды}
Протолитическая реакция -- реакция переноса протона. \\
Рассмотрим некоторый растворитель $AH$ (где $А$- некоторый, не обязательно одноатомный анион), имеющиий на протоне некоторый положительный заряд ($\delta^+$). При этом основное взаимодействие $A-H$ несколько ослабеет и становится возможнным перенос протона между двумя молекулами, участвующими в образовании водородной связи. Протекающие процессы могут быть описаны уравнениями:

$$A-H + A-H \rightleftarrows A-H---A-H \rightleftarrows A---H-A-H \rightleftarrows A^- + H_2A^+$$
Вновь образовавшая избыточная связь $A-H$ оказывается слабее исходной. Поэтому  равновесие реакции существенно смещено влево и константа равновесия мала, однако некоторый выигрыш в энтропии обеспечивает протекание такого рода процессов в любых не апротонных растворителях. Это явление получило название автопротолиза. Характерным примером такового является равновесие, устанавливающееся в водном растворе:

$$H_2O + H_2O \rightleftarrows H_3O^+ + OH^-$$
Автопротолизом обусловлена маленькая, но существующая, электропроводимость чистой воды. Поскольку концентрация воды величина постоянная, константа равновесия может быть выражена соотношением: 
$$K_{\alpha} = \left[H_3O^+\right]\left[OH^-\right]$$
\ul{Константа автопротолиза является ионным произведением воды.}\\
Константа автопротолиза увеличивается  с ростом температуры. Это соотношение оказывается справедливым также для любых растворов кислот, солей и оснований, поскольку концентрация воды во вспх этих системах меняется сравнительно слабо и близка к 55,5 моль/л. При растворении таких веществ в воде меняется ионная сила раствора, что приводит к величинам коэффициентов активности для катионов, отличных от единицы.

\subsection*{Сильные и слабые кислоты. Факторы, определяющие силу кислот}

Направление смещения кислотно-основного равновесия:\\
Кислотно-основные равновесия смещены в сторону более слабой кислоты и более слабого основания.\\\\
Кислота тем сильнее, чем легче она отдает протон, а основание тем сильнее, чем легче оно принимает протон и прочнее его удерживает. Молекула (или ион) слабой кислоты не склонна отдавать протон, а молекула (или ион) слабого основания не склонна его принимать, этим и объясняется смещение равновесия в их сторону.\\
Силу кислот, а также силу оснований можно сравнивать только в одном и том же растворителе
Так как кислоты могут реагировать с разными основаниями, то соответствующие равновесия будут смещены в ту или иную сторону в разной степени. Поэтому для сравнения силы разных кислот определяют, насколько легко эти кислоты отдают протоны молекулам растворителя. Аналогично определяется и сила оснований. \\\\
Сила кислоты – характеристика кислоты, показывающая, насколько легко кислота отдает протоны молекулам данного растворителя. \\
Сила основания – характеристика основания, показывающая, насколько прочно основание связывает протоны, оторванные от молекул данного растворителя. \\\\
И кислоты, и основания можно сравнивать между собой по силе в водных растворах. В одном и том же растворителе сила кислоты в значительной степени зависит от энергии рвущейся связи А-Н, а сила основания – от энергии образующейся связи $B-H$. \\
Для количественной характеристики силы кислоты в водных растворах можно использовать константу кислотно-основного равновесия обратимой реакции данной кислоты с водой: 
$$K = \frac{\left[A^-\right]\left[H_3O^+\right]}{\left[HA\right]\left[H_2O\right]}$$
Для характеристики силы кислоты в разбавленных растворах, в которых концентрация воды практически постоянна, пользуются константой кислотности: 
$$K_k = \frac{\left[A^-\right]\left[H_3O^+\right]}{\left[HA\right]}$$
Совершенно аналогично для количественной характеристики силы основания можно использовать константу кислотно-основного равновесия обратимой реакции данного основания с водой:
$$K = \frac{\left[HA\right]\left[OH^-\right]}{\left[A^-\right]\left[H_2O\right]}$$
а в разбавленных растворах – константу основности: 
$$K_o = \frac{\left[HA\right]\left[OH^-\right]}{\left[A^-\right]}$$
Основание тем сильнее, чем слабее сопряженная кислота. И наоборот, кислота тем сильнее, чем слабее сопряженное основание. \\\\
Чем определяется сила кислоты? \\
Возьмем случай, когда молекула электронейтральна. \\
В первую очередь, соотношением донорных свойств неподеленной электронной пары и склонности к диссоциации связи Э-H. \\
Активность первой падает, а второй - возрастает при продвижении по периоду таблицы Менделеева слева-направо. Это определяется ростом электроотрицательности связанного с протоном атома, а во-вторых тем, что донорная способность электронных пар тем выше, чем меньше их число. В этом же ряду понижается электронная плотность связи Э-H, что облегчает ее диссоциацию. \\
При продвижению сверху вниз по подгруппе донорные свойства электронной пары убывают в связи с уменьшением "склонности к $sp^3$ - гибридизации". Это явление можно объяснить тем, что ввиду увеличения размера орбиталей и понижения разницы между энергией последней заполненной (p) и последующей вакантной (чаще всего d) орбитали электронная плотность этой пары оказывается существенно более делокализованной. С другой стороны, с ростом радиуса элемента прочность связи Э-H существенно убывает.

\subsection*{Концентрация ионов водорода. pH.}

Водные растворы могут быть нейтральными, кислыми или щелочными. В кислых растворах содержится избыток ионов $H^+$, а в щелочных – избыток ионов $OH^-$. В нейтральных растворах количество этих ионов всегда одинаково и при этом чрезвычайно мало – по $10^{-7}$ моль/л каждого иона. Низкая концентрация ионов $H^+$ и $OH^-$ в нейтральном растворе вполне объяснима – ведь эти ионы охотно реагируют друг с другом, поскольку в результате образуется прочное, малодиссоциированное соединение $H_2O$. Таким образом, в нейтральном растворе присутствуют только те ионы $H^+$ и $OH^-$, которые образовались из самой воды естественным путем, в результате ее обратимой диссоциации.

Для воды и ее растворов при неизменной температуре произведение концентраций ионов водорода и гидроксид-ионов есть величина постоянная. Эта постоянная величина называется ионным произведением воды $K_w$

$$K_w = \left[H^+\right]\left[OH^-\right] = 10^{-14}$$

Водородный показатель показывает, что кислотность или щелочность растворов можно выразить через концентрацию одних только ионов водорода $H^+$.

Водородный показатель определяется следующим образом: рН раствора равен обратному логарифму от концентрации ионов водорода в этом растворе.

$$pH = -\lg\left[H^+\right]$$

\subsection*{Гидратированные ионы, как пример слабых кислот.}


По определению Льюиса, катион любого элемента склонен к образованию координационной связи с электронными парами лигандов и, следовательно, является кислотой.

Действительно, в водных растворах катионы координируют вокруг себя молекулы воды.При этом за счет образования координационных связей, электронная плотность с последних смещается к металлу. Это приводит к увеличению положительного заряда на протонах воды и повышает веротность кислотной диссоциации по следующей схеме:

$$\left[(H_2O)_n MOH_2\right]^{z+} + OH_2 \rightleftarrows \left[(H_2O)_n MOH-H---OH_2\right]^{z+} \rightleftarrows $$
$$ \rightleftarrows  \left[(H_2O)_n MOH\right]^{z-1} + H_3O^+$$

\section*{Вопрос №16}
Гидролиз -- это взаимодействие соли с водой с образованием кислоты и основания.\\
В результате гидролиза изменяется $pH$ среды. \\
Степень гидролиза ($\beta$) -- это отношение концентрации гидролизовавшихся продуктов к исходной. 
$$ \beta = \dfrac{C}{C_0}  $$

\subsection*{Гидролиз солей, образованных сильной кислотой и слабым основанием. Константа и степень гидролиза}
Рассмотрим для примера гидролиз $NH_4Cl$. При диссоциации образуются: $NH_4^+ $ и $Cl^-$, так как $Cl^-$ сопряжен с сильной кислотой -- он не склонен к гидролизу. Ион аммония:
$$ NH_4^+ + H_2O \leftrightarrow NH_3 + H_3O^+  $$
Образующиеся ионы по мере увеличения концентрации будут взаимодействовать друг с другом с образованием исходных веществ, тогда константа равновесия этой реакции:
$$ K_{\text{г}} = \dfrac{[NH_3][H_3O^+]}{[NH_4^+]} $$
Очевидно, что $ K_{\text{г}}$ -- константа диссоциации сопряженной с гидроксидом аммония кислоты. Пусть степень гидролиза равна $\beta$, тогда  из стехиометрии:
$$ C = [NH_3] = [H_3O^+] = \beta C_0 $$
Концентрация оставшихся $NH_4^+$ составляет $(1-\beta)C_0$, отсюда:
$$ K_{\text{г}} = \dfrac{\beta^2 C_0^2}{(1-\beta)C_0} = \dfrac{\beta^2C_0}{(1-\beta)} $$
Поскольку величина гидролиза обычно мала и $(1-\beta)C_0 \approx 1 $, то $ \beta \approx \sqrt{\frac{K}{C_0}}$. \\
Домножим числитель и знаменатель для $ K_{\text{г}} $ на $[OH^-] $:
$$ K_{\text{г}} = \dfrac{[NH_3][H_3O^+][OH^-]}{[NH_4^+][OH^-]} = \dfrac{K_a[NH_3]}{[NH_4^+][OH^-]} = \dfrac{K_a}{K_o}   $$
Т.о. $ K_{\text{г}}$ соли, образованной сильной кислотой и слабым основанием, равна отношению константы автопротолиза растворителя к константе диссоциации соответствующего основания в нем. \\ \\
Степень гидролиза увеличивается с понижением концентрации раствора. \\
Степень гидролиза растет с повышением температуры (для веществ с малой степенью гидролиза, т.к. $\Delta{H} > 0$). \\
Гидролиз можно подавить добавлением продуктов гидролиза. (в данном случае: $NH_3$ или сильной кислоты (т.к. она станет основным донором $H_3O^+$)).
 
\subsection*{Гидролиз солей, образованных слабой кислотой и сильным основанием}
Аналогично рассуждению выше, только константа гидролиза соли, образованной слабой кислотой и сильным основанием, равна отношению константы автопротолиза растворителя к константе диссоциации сопряженной с анионом кислоты. \\
Влияние разбавления раствора и повышения температуры на степень гидролиза аналогично. \\
Также  можно связать продукты гидролиза для повышения степени гидролиза (например добавлением сильной кислоты, которая будет связывать $OH^-$)\\
\ul{Если соль образована многоосновной слабой кислотой и сильным основанием или сильной кислотой и слабым многоосновным основанием, гидролиз обычно останавливается на первой ступени.} \\
\subsection*{Гидролиз солей слабых кислот и оснований}
Для соли $AX$, образованной слабой кислотой и слабым основанием, гидролиз протекает одновременно и по катиону, и по аниону:
$$ X^- + H_2O \leftrightarrow HX + OH^- $$
$$ A^+ + 2H_2O \leftrightarrow AOH + H_3O^+ $$
При этом выделяющиеся $OH^-$ и $H_3O^+$ взаимодействуют с образованием воды и равновесие смещается вправо. Поэтому гидролиз таких солей протекает в существенно большей степени, по сравнению с ранее рассмотренными случаями. В случае, если хотя бы один продукт гидролиза удаляется из сферы реакции (в виде газа или осадка), гидролиз протекает практически нацело:
$$ Al_2S_3 + 6H_2O = 2Al(OH)_3 \downarrow + 3H_2S \uparrow $$
$$ ZrCl_4 + H_2O = ZrOCl_2 + 2HCl \uparrow $$

При гидролизе соли слабого основания и кислоты  $[HX]$ и $[AOH]$ $\gg$ $[OH^-] $ и $[H_2O^+] $, которые нейтрализуются с образованием воды.
$$ X^- + A^+ + H_2O \leftrightarrow HX + AOH $$
Поэтому $[HX] \approx [AOH]$. Пусть концентрация соли -- $C_0$, степень гидролиза -- $\beta$, а константы диссоциации кислоты и основания --  $K_k, K_o $:
$$ K_{\text{г}} = \dfrac{[HX][AOH]}{[X^-][A^+]} $$
Домножим на ионное произведение воды:
$$ K_{\text{г}} = \left(\dfrac{[HX]}{[X^-][H_3O^+]}  \right) \left(\dfrac{[AOH]}{[A^+][OH^-]}  \right) \left([H_3O^+][OH^-]\right) = \dfrac{K_a}{K_kK_o} $$
Следовательно:
$ [HX] = [AOH] = \beta C_o $, а $ [X^-]=[A^+] = (1-\beta)C_0 $, тогда:
$$ K_{\text{г}} = \left[\dfrac{\beta C_0}{(1-\beta)C_0}\right]^2 = \left[\dfrac{\beta}{(1-\beta)}\right]^2  $$
Т.о. степень гидролиза такой соли не зависит от концентрации, она определяется отношением:
$$ \sqrt{K_{\text{г}}} = \dfrac{\beta}{(1-\beta)} \Rightarrow \beta = \dfrac{\sqrt{K_{\text{г}}}}{[1+\sqrt{K_{\text{г}}}]}   $$
А $pH$ для такого раствора:
$$ pH = -\dfrac{1}{2} \lg{\left(\dfrac{K_aK_k}{K_o} \right)} = \dfrac{1}{2}(pK_a + pK_k - pK_o) $$
Т.е. $pH$ таких растворов определяется соотношением $pK_k $ и $pK_o $.

\subsection*{Образование кластеров при гидролизе}
Гидроксид-ионы в достаточно концентрированных растворах солей полизарядных катионов, подверженных гидролизу, могут выступать в качестве связующих мостиков между ними. При этом происходит образование полиатомных кластеров. \\
Так, цирконил-ионы в растворе представляют собой комплексы, в которых четыре катиона, расположенные в вершинах квадрата, соединены друг с другом за счет пар гидроксид-ионов. Кроме того, каждый атом достраивает свой свой координационный полиэдр четырьмя молекулами воды.
\begin{figure}[H]
	\centering
	\includegraphics[scale=1]{cluster}
	\caption{}
	\label{}
\end{figure}
\subsection*{Буферные растворы}
Буферный раствор - это раствор, содержащий равновесную систему, способную поддерживать практически постоянное значение рН при разбавлении или при добавлении небольших количеств кислоты или щелочи. \\\\
Если в одном и том же растворе присутствуют в сопоставимых концентрациях слабая кислота $HA$ и ее соль с сильным основанием $KA$, то гидролиз соли и диссоциация кислоты подавляются. Поэтому:
$ [HA] = C_{}HA, [A^-] = C_{KA} $, тогда из выражения константы диссоциации кислоты имеем:
$$ K = \dfrac{[H_3O^+][A^-]}{[HA]}, [H_3O^+] = \dfrac{KC_{HA}}{C_{KA}} $$
$$ pH = pK + p \left( \dfrac{C_{HA}}{C_{KA}} \right) $$
Добавим к этому раствору некоторое количество сильной кислоты, составляющее $10\%$ от концентраций исходных веществ, при этом десятая часть соли $KA$ перейдет в кислоту $ HA$ и $pH$ раствора изменится на $p(\frac{1,1}{0,9}) = 0,087$ единиц.

\section*{Вопрос №17}
\subsection*{Термодинамические аспекты получения абсолютно чистых веществ}
Процесс очистки происходит тем полнее, чем ближе к равновесным условиям он проводится. \\
Рассмотрим термодинамику процесса загрязнения вещества: \\
Внедрение в вещество некоторой посторонней примеси чаще всего приводит к повышению энтальпии. С хорошей точностью -- полигрыш в энтальпии пропорционален числу посторонних атомов на моль вещества. \\
Также введение примеси приводиь к увеличению энтропии. В первую очередь возрастает конфигурационная энтропия:
$$ S = k \ln W $$
где $W$ - число возможных способов размещения атомов в узлах кристаллической решетки. Для чистого вещества конфигурационная энтропия равна $0$.Если появляется один примесный атом, то число способов его размещения в $N$ узлах решетки равно $N$.\\
Это достаточно большая величина, для ее компенсации при комнатной температуре необходимо, чтобы:
$$ \Delta{H} = Nk\ln{N} \cdot T = 6,02\cdot 10^{23} \cdot 1,38 \cdot 10^{-23} \ln{(6,02\cdot 10^{23})} \cdot 298 = 136 \text{ кДж/моль}$$  
При внедрении следующего атома прирост энтропии уже меньше: первый атом можно разместить по $N$ позициям, второй -- по $N-1$ и т.д. до замещения половины атомов основного вещества. \\
Графики будут выглядеть так:
\begin{figure}[H]
	\centering
	\includegraphics[scale=0.4]{purification}
	\caption{}
	\label{}
\end{figure}
При равновесии должно соблюдаться соотношение:
$$\Delta{H} - T\Delta{S} = 0 $$
\subsection*{Методы очистки твердых и жидких веществ}
\begin{figure}[H]
	\centering
	\includegraphics[scale=0.6]{methods}
	\caption{}
	\label{}
\end{figure}
\subsection*{Перекристаллизация из раствора (связь с $TX$-диаграммой системы сольвода), коэффициент распределения примесей}
Этот метод применяется для двух и более твердых веществ (чаще всего солей), находящихся в смеси друг с другом. \\
Если один из компонентов нерастворим, то необходимо растворить второй и тогда разделить с помощью фильтрации, декантации и т.п. \\
Также можно выбрать количество растворителя так, чтобы в осадке остался один компонент (точнее, его часть) и разделить фильтрацией. \\
Но чаще используют способ, связанный с изменением растворимости веществ с ростом/понижением температуры. \\
Пусть имеется $m$ грамм слабозагрязненного вещества $B$, растворимость которого \ul{увеличивается} с температурой и при $T_1$ составляет $c_1$, а при $T_2$ -- $c_2$ г на $100$ г растворителя. Тогда необходимое для насыщения по $B$ при $T_1$ количество растворителя составит: $100 \frac{m}{c_1}$. При $T_2$ в нем останется $m\frac{c_2}{c_1}$ г $B$. Т.е. выход после перекристаллизации составит:
$$ m - m\dfrac{c_2}{c_1} = m\dfrac{1-c_2}{c_1} $$
Если растворимость уменьшается с ростом $T$, то насыщение проводят при более низкой $T$. \\
Если растворимость слабо меняется с температурой, используют упаривание раствора.\\



\subsection*{Соосаждение, адсорбция, окклюзия}




\section*{Вопрос №18}
\subsection*{Кристаллизация из расплава (ТХ-диаграммы)}
Диаграмма плавкости веществ с неограниченной растворимостью в жидком и полной нерастворимостью в твердом состоянии.

Этот тип диаграмм характерен для веществ, заметно отличающихся структурой кристаллов.

Диаграмма температура -– состав строится на основании кривых охлаждения (нагревания). Кривые охлаждения – графическое изображение зависимости температуры от времени для исходных чистых веществ A и B и их смесей различного состава. Вид этих кривых свидетельствует о наличии или отсутствии фазовых превращений при некоторых определенных температурах или в интервале температур. 
\begin{figure}[H]
\centering
\includegraphics[scale=.45]{cristallization-diagram.png}
\caption{}
\label{}
\end{figure}

Весьма часто твердая фаза, выделяющаяся при охлаждении расплавов, состоит из кристаллов, образуемых обоими компонентами. Такая однородная система имеет переменный состав и называется твердым раствором. Твердые растворы – системы однофазные, подобно обычным жидким растворам, но в отличие от последних имеют кристаллическую структуру.

\begin{figure}[H]
\centering
\includegraphics[scale=.600]{cristallization-diagram2.png}
\caption{}
\label{}
\end{figure}

Неограниченной взаимной растворимостью в твердом состоянии обладают вещества, имеющие близкие значения атомных или ионных радиусов, энергии химической связи, сходное строение электронных оболочек и одинаковый тип кристаллической решетки (изоморфные вещества). Примерами таких систем могут служить $Au-Ag$, $Cu-Au$, $Se-Ge$, $NaCl-NaBr$ и другие.

\subsection*{Зонная плавка}

Зонная плавка (зонная перекристаллизация) - метод очистки твёрдых веществ, основанный на различной растворимости примесей в твёрдой и жидкой фазах. Метод является разновидностью направленной кристаллизации, от которой отличается тем, что в каждый момент времени расплавленной является некоторая небольшая часть образца. Такая расплавленная зона передвигается по образцу, что приводит к перераспределению примесей. Если примесь лучше растворяется в жидкой фазе, то она постепенно накапливается в расплавленной зоне, двигаясь вместе с ней. В результате примесь скапливается в одной части исходного образца. По сравнению с направленной кристаллизацией этот метод обладает большей эффективностью. Метод был предложен Уильямом Гарднером Пфанном в 1952 году и с тех пор завоевал большую популярность. В настоящее время метод используется для очистки более 1500 веществ.

Распределение примеси характеризуется коэффициентом распределения, который равен

$$ K={\frac {C_{S}}{C_{L}}}$$

где $C_S$ - концентрация примеси в твёрдой фазе, $C_L$ - концентрация примеси в жидкой фазе.

Иногда вместо коэффициента распределения $K$ используют коэффициент разделения $\alpha$, который равен

$$\alpha ={\frac {C_{S}(1-C_{L})}{C_{L}(1-C_{S})}}$$

Примеси, для которых коэффициент распределения K < 1, концентрируются в расплавленной зоне и вместе с ней перемещаются к концу слитка. С другой стороны от расплавленной зоны образуются слои вещества, более чистого относительно примесей, для которых K < 1. Те примеси, для которых K > 1, наоборот, концентрируются в начале слитка. Если осуществить многократное прохождение расплавленной зоны, то примеси с K < 1 соберутся в конце слитка. Для примесей с К > 1 метод мало эффективен. Самые чистые части слитка (из середины) используются для изготовления приборов. Таким методом можно очистить германий до образцов с удельным сопротивлением порядка 70 Ом·см, в которых остаётся примерно один атом примеси на 1010 атомов германия.

Если расплав вступает в реакцию с материалом тигля (лодочки), или очищаемое вещество имеет высокую температуру плавления ($>1500^o C$), применяют бестигельную зонную плавку.

Метод обладает рядом недостатков. Основной недостаток - невозможность масштабирования, так как скорость процесса определяется скоростью диффузии примеси. Поэтому метод применяется для конечной стадии очистки при получении особо чистых веществ. Максимальные габариты лодочки - длина 50 см, толщина - 2-3 см, длина расплавленной зоны - 5 см.
\subsection*{ТХ-диаграмма системы жидкость-пар, дистилляция}
TX-диаграмма системы жидкость-пар изображена ниже:
\begin{figure}[H]
\centering
\includegraphics[scale=1.50]{vapor-diagram.png}
\caption{}
\label{}
\end{figure}

Две линии на ней - начало(снизу) и конец(сверху) кипения. Внизу - жидкость, посередине - равновесие жидкости и пара, сверху - пар. 

\textbf{Дистилляция} - перегонка, испарение жидкости с последующим охлаждением и конденсацией паров. Дистилляцию рассматривают прежде всего как технологический процесс разделения и рафинирования многокомпонентных веществ - в ряду других процессов с фазовым превращением и массообменом: сублимация, кристаллизация, жидкостная экстракция и некоторых других. 

Различают дистилляцию с конденсацией пара в жидкость (при которой получаемый дистиллят имеет усреднённый состав вследствие перемешивания) и дистилляцию с конденсацией пара в твёрдую фазу (при которой в конденсате возникает распределение концентрации компонентов). 

Продуктом дистилляции является дистиллят или остаток (или и то, и другое) - в зависимости от дистиллируемого вещества и целей процесса. Основными деталями дистилляционного устройства являются обогреваемый контейнер (куб) для дистиллируемой жидкости, охлаждаемый конденсатор (холодильник) и соединяющий их обогреваемый паропровод.

Существуют разные вариации на тему, вроде перегонки с водяным паром, но это экзотика, которая врядли кому-то нужна.


\subsection*{Возгонка, РТ-диаграммы перегоняемых веществ}

Обратным процессом является десублимация. Примером десублимации являются такие атмосферные явления, как иней на поверхности земли и изморозь на ветвях деревьев и проводах.

На диаграмме состояний (де)сублимация - это переход линии, соединяющей абсолютный ноль и тройную точку.

\begin{figure}[H]
\centering
\includegraphics[scale=.50]{sublimation.jpg}
\caption{}
\label{}
\end{figure}

Сублимацию применяют в химии для очистки некоторых веществ, например иода и бензойной кислоты.

Схема установки для сублимации изображена ниже.
\begin{figure}[H]
\centering
\includegraphics[scale=1.3]{sublimation2.png}
\caption{}
\label{}
\end{figure}

\begin{enumerate}
\item вход воды для охлаждения
\item выход воды для охлаждения
\item выход для вакуума
\item сублимационная камера
\item продукт
\item очищаемое вещество
\end{enumerate}


\subsection*{Транспортные реакции}
Этот метод широко используется при получении особо чистых веществ для полупроводниковой техники и радиоэлектроники. Принцип его состоит в том, что очищаемое твердое или жидкое вещество $А$, взаимодействуя по обратимой реакции с газообразным веществом $В$, образует газообразный продукт $С$, переносимый (транспортируемый) в другую часть системы, где вследствие изменения условий происходит его разложение с выделением чистого вещества $А$:
$$A + B \rightleftarrows C$$

Классическим примером транспортной реакции является очистка металлического никеля через его карбонил (метод Монда). Порошок никеля обрабатывают при $45-50 ^oС$ окcидом углерода:
$$Ni + 4CO \rightleftarrows \left[Ni(CO)_4\right]$$

Газообразный $[Ni(CO)_4]$ поступает в другую часть реакционного аппарата, где при $180-200 ^oС$ разлагается, давая чистый никель, а $СО$ снова направляют в процесс.

Метод транспортных реакций применяется для получения различных чистых веществ как простых, так и сложных. В качестве транспортирующего агента часто используют галогены, галогеноводороды, водяной пар, кислород, водород и др. Например, при получении особо чистых $Ni$, $Cu$, $Fе$, $Cr$, $Si$, $Ti$, $Hf$, $Th$, $V$, $Nb$, $Та$ и $U$ применяют иод.

Направление транспорта (из зоны с низкой температурой в зону с высокой температурой или наоборот) определяется термодинамическими свойствами (знаком теплового эффекта).

При экзотермических реакциях транспорт вещества производится в более нагретую зону, как в приведенном примере с очисткой Ni. Метод транспортных реакций удобен для очистки от элементов, отличающихся по своим химическим свойствам от основного элемента.

Для глубокой очистки от элементов-аналогов он мало пригоден. Достоинством транспортных реакций является возможность проведения всех операций в стерильных условиях, поскольку эти реакции проходят в замкнутом объеме и без больших количеств реагентов.

\subsection*{Хроматография и адсорбция. Экстракция. Ионный обмен.} 
\textbf{Хроматография} -  метод разделения и анализа смесей веществ, а также изучения физико-химических свойств веществ. Основан на распределении веществ между двумя фазами - неподвижной (твёрдая фаза или жидкость, связанная на инертном носителе) и подвижной (газовая или жидкая фаза, элюент). Название метода связано с первыми экспериментами по хроматографии, в ходе которых разработчик метода Михаил Цвет разделял ярко окрашенные растительные пигменты. Опыт цвета показан ниже - он делил хлорофиллы, ксантины и что-то еще в этом духе.

\begin{figure}[htp]
\centering
\includegraphics[scale=.5]{chromatorgamm.jpg}
\caption{}
\label{}
\end{figure}

Существует дикое количество видов, по агрегатному состоянию фаз, движущей силе(электрохроматография, ВЭЖХ и т.д.), механизму сорбции(хемосорбция, молекулярные сита, и т.д.), форме оборудования(тонкослойная, колоночная, капиллярная), но не думаю, что их от нас хотят все услышать, тем более, что идея везде одна.

\textbf{Адсорбция} -  самопроизвольный процесс увеличения концентрации растворённого вещества у поверхности раздела двух фаз (твёрдая фаза - жидкость, конденсированная фаза - газ) вследствие нескомпенсированности сил межмолекулярного взаимодействия на разделе фаз. Адсорбция является частным случаем сорбции, процесс, обратный адсорбции - десорбция.

Поглощаемое вещество, ещё находящееся в объёме фазы, называют адсорбтив, поглощённое - адсорбат. В более узком смысле под адсорбцией часто понимают поглощение примеси из газа или жидкости твёрдым веществом (в случае газа и жидкости) или жидкостью (в случае газа) - адсорбентом. При этом, как и в общем случае адсорбции, происходит концентрирование примеси на границе раздела адсорбент-жидкость либо адсорбент-газ. Процесс, обратный адсорбции, то есть перенос вещества с поверхности раздела фаз в объём фазы, называется десорбция. Если скорости адсорбции и десорбции равны, то говорят об установлении адсорбционного равновесия. В состоянии равновесия количество адсорбированных молекул остается постоянным сколько угодно долго, если неизменны внешние условия (давление, температура и состав системы).

\textbf{Экстракция} - это извлечение вещества из раствора или сухой смеси с помощью растворителя (экстрагента), практически не смешивающегося с исходной смесью.

Экстракция может быть разовой (однократной или многократной) или непрерывной (перколяция).

Простейший способ экстракции из раствора - однократная или многократная промывка экстрагентом в делительной воронке. Делительная воронка представляет собой сосуд с пробкой и краном для слива нижнего слоя жидкости. Для непрерывной экстракции используются специальные аппараты - экстракторы, или перколяторы.

Для извлечения индивидуального вещества или определённой смеси (экстракта) из сухих продуктов в лабораториях широко применяется непрерывная экстракция по Сокслету.

В лабораторной практике химического синтеза экстракция может применяться для выделения чистого вещества из реакционной смеси или для непрерывного удаления одного из продуктов реакции из реакционной смеси в ходе синтеза.

Экстракция применяется в химической, нефтеперерабатывающей, пищевой, металлургической, фармацевтической и других отраслях, в аналитической химии и химическом синтезе.

\textbf{Ионный обмен} - это обратимая химическая реакция, при которой происходит обмен ионами между твердым веществом (ионитом) и раствором электролита. Ионный обмен может происходить как в гомогенной среде (истинный раствор нескольких электролитов), так и в гетерогенной, в которой один из электролитов является твёрдым (при контакте раствора электролита с осадком, ионитом и др.).

\emph{Катионный обмен} - частный случай ионного обмена, под которым в химии понимают обратимый процесс стехиометрического обмена ионами между двумя контактирующими фазами.

Ионный обмен используется в химии для замены одного иона на другой, с тем же знаком заряда.
\subsection*{Коэффициент распределения.}

Коэффициент разделения (коэффициент распределения) - концентрационная характеристика фазового превращения или фазового равновесия двух- или многокомпонентного вещества. Термин введен около 1950 г. для рассмотрения процессов с фазовым превращением и массообменом (дистилляция, сублимация, кристаллизация, жидкостная экстракция и некоторые другие) как технологических процессов разделения и рафинирования двух- и многокомпонентных веществ. В первую очередь рассматриваются так называемые равновесный, кинетический и эффективный коэффициенты разделения (распределения).

$$K = \frac{C_A^1}{C_A^2}$$
$C_A^1$ - равновесная концентрация вещества в первой фазе\\
$C_A^2$ - равновесная концентрация вещества во второй фазе\\
$K$ - коэффициент распределения

Написанное выше верно для идеального случая - когда фазы в равновесии, то есть прошло бесконечное время. В реальности исползуется коэффициент, в котором не равновесные, а реальные - экспериментальные или литературные концентрации.


\section*{Вопрос № 22}
\subsection*{Электролиз растворов и расплавов}
Электролиз — это окислительно-восстановительный процесс, протекающий на электродах (катоде($-$) и аноде ($+$))
при прохождении электрического тока через расплав или раствор электролита. \\
\begin{itemize}
	\item Электролиз водных растворов: \\
	Продукты, выделяющиеся на электродах, зависят от
	природы ионов, находящихся в растворе. Восстанавливаемый на катоде продукт
	определяется стандартным электродным потенциалом металла (его положением в
	ряду стандартных электродных потенциалов). \\
	\ul{Катионы}: \\
	\begin{itemize}
		\item Если металл стоит до $H$, то вместо него электролизу подвергается вода:
		$$ 2H_2O + 2e = H_2 + 2OH^- $$
		\item Катионы металлов, стоящие в ряду напряжений после алюминия до водорода, могут восстанавливаться вместе с молекулами воды:
		$$ 2H_2O + 2e = H_2 + 2OH^- $$
		$$ Zn^{2+} + 2e = Zn^0 $$
		\item Если металл стоит после Н, то он сам восстанавливается:
		$$ Cu^{2+} + 2e = Cu^0 $$
		Медь осаждается на катоде.
	\end{itemize}
	\ul{Анионы:} \\
	\begin{itemize}
		\item Кислородсодержащие кислотные остатки -- вместо них электролизу подвергается вода:
		$$ 2H_2O - 4e = O_2 + 4H^+ $$ 
		\item Бескислородные кислотные остатки — окисляются до простого вещества:
		$$ Cl^- - 1e = Cl^0 $$
		Исключение: $F^-$, вместо него будет выделяться кислород, так как растворы фторидов имеют существенно больший потенциал окисления, нежели кислород.
	\end{itemize}
	\item Электролиз расплавов:\\
	Т.к. если потенциал пары (раствор соли)/металл выше $0,413V$ (что представляет собой потенциал $2H^+/H_2 $), на катоде будет выделяться водород $\Rightarrow$ активные металлы нельзя выделить электролизом раствора. Поэтому для их выделения используют электролиз расплава. \\
	Механизм тот же, но плюсом будет отсутствие конкурирующих процессов выделения кислорода и водорода из воды.
\end{itemize}
\subsection*{Источники тока}
Химический источник тока — источник ЭДС, в котором энергия протекающих в нём
химических реакций непосредственно превращается в электрическую энергию.
\subsection*{Гальванические элементы}

Принцип действия элемента основан на взаимодействии двух металлов через электролит, приводящем к возникновению в замкнутой цепи электрического тока.\\
Его недостатками являются:
\begin{itemize}
	\item присутствие жидкости;
	\item малое выходное напряжение; 
	\item малая емкость;
\end{itemize}
\begin{figure}[H]
	\includegraphics[scale=1]{galel.jpg}
	\centering
	\caption{Простейший источник тока}
\end{figure}
Реакции на электродах для данного гальванического элемента:
$$ Zn - 2e = Zn^{2+} $$
$$ MnO_2 + 4NH_4^+ + 2e = Mn^{2+} + H_2O + 4NH_3 $$
Из-за необратимости реакций, протекающих в гальваническом элементе, его нельзя перезарядить.\\
Так, цинковые батарейки характеризуются низким выходным напряжением и их нельзя перезарядить.
\subsection*{Аккумуляторы}
Аккумуляторы — источники многоразового действия, в которых химические реакции,
непосредственно превращаемые в электрическую энергию, многократно обратимы.
Среди преимуществ: 
\begin{itemize}
	\item Напряжение больше, чем у гальванических элементов (жидкие электролиты);
	\item Возможность перезарядки;
	\item Возможность использовать твердые электролиты.
\end{itemize}
Из недостатков: в свинцовом аккумуляторе используется агрессивный электролит (серная кислота), а с твердым электролитом -- низкое напряжение.
\subsection*{Топливные элементы}
\begin{figure}[H]
	\includegraphics[scale=0.8]{ttt.jpg}
	\centering
	\caption{Устройство топливного элемента}
\end{figure}
Кислород и водород в топливных элементах разделяются прослойкой материала (мембраной) с проводимостью по ионам водорода и с нанесенными с двух сторон слоями катализатора. Ионизируясь и оставляя электроны на аноде, водород диффундирует к катоду, вступая в реакцию с кислородом. Происходит реакция окисления водорода кислородом с образованием воды.\\
Минусы топливных элементов:
\begin{itemize}
	\item Низкая электропроводность при комнатной температуре и низкой влажности;
	\item Высокая стоимость платинового катализатора и возможность его отравления даже следами $CO$, который почти всегда есть в дешевых формах водорода.
\end{itemize}

\section*{Вопрос №24}

\subsection*{Скорость химической реакции}

Многое удается узнать о химических реакциях, изучая скорость их протекания и факторы, от которых она зависит. Этим занимается раздел химии, называемый химической кинетикой.

\textbf{Скоростью химической реакции} называется количество вещества, вступающего в реакцию или образующегося при реакции за единицу времени в единице объема системы.

$$V = \frac {dC}{dt}$$

Количество вещества выражают в \emph{молях}, время как правило в \emph{секундах} а объем в \emph{литрах}.

Таким образом, скоростью реакции называют изменение концентрации какого-нибудь вещества, участвующего в реакции, за единицу времени (например, за секунду или за минуту). Отсюда другое определение скорости реакции:

Следовательно, размерность у скорости реакции такая: "моль/л · сек".

За скоростью реакции $$A + B \Rightarrow C$$ можно следить по расходованию одного из реагентов ($A$ или $B$), либо по накоплению продукта ($C$). Здесь мы сталкиваемся с серьезной проблемой: скорость реакции может постоянно изменяться. Действительно, в начале реакции, когда молекул $A$ и $B$ еще много, столкновения между ними происходят гораздо чаще, чем в конце реакции, когда молекул $A$ и $B$ уже намного меньше. Как мы знаем, столкновения молекул являются поводом для реакции между ними. 

\subsection*{Закон действующих масс и константа скорости} 

$$aA + bB \Rightarrow cC$$

$$V = k \left[A\right]^a\left[B\right]^b$$


Скорость реакции прямо пропорциональна произведению концентраций всех реагентов в степенях, равных \emph{порядку реакции} поэтому реагенту. В случае элементарной реакции, порядок равен коэффициенту перед веществом. Коэффициент пропорциональности - константа скорости. Ее размерность зависит от порядка реакции. Выведем кинетические уранения для нулевого, первого и энного порядка. $C^*$ - константа интегрирования, $C_0$ - начальная концентрация вещества. 

\subsection*{Нулевой порядок}
$$\frac{dC}{dt} = -kC^0 = -k$$
$$\int dC = -\int kdt$$
$$C = -kt + C^*$$
$$C = C_0 - kt$$

\subsection*{Первый порядок}
$$\frac{dC}{dt} = -kC^1 = -kC$$
$$\int \frac{dC}C = -\int kdt$$
$$ln C = -kt + C^*$$
$$C = e^{-kt}\cdot e^{C^*}$$
$$C = C_0e^{-kt}$$

\subsection*{$n$-ый порядок}
$$\frac{dC}{dt} = -kC^n = -kC$$
$$\int \frac{dC}C^n = -\int kdt$$
$$-\frac{1}{(n-1)C^{n-1}} = -kt + C^*$$
$$-\frac{1}{(n-1)C^{n-1}} = -kt - \frac{1}{(n-1)C_0^{n-1}}$$
$$ \frac 1{C^{n-1}} = +kt(n-1) + \frac1{C_0^{n-1}}$$

\subsection*{Связь кинетики и константы равновесия}

Что такое равновесие? Это, с точки зрения кинетики, равенство скоростей прямой и обратной реакций. Запишем это:
$$aA + bB \Rightarrow cC + dD$$
$$K =\frac{\left[C\right]^c\left[D\right]^d}{\left[A\right]^a\left[B\right]^b}$$
$$V_1 = k1 \left[A\right]^a\left[B\right]^b$$
$$V_{-1} = k_{-1}\left[C\right]^c\left[D\right]^d$$
$$\frac{V_1}{V_{-1}} = \frac{k1 \left[A\right]^a\left[B\right]^b}{k_{-1}\left[C\right]^c\left[D\right]^d} = 1$$
$$\frac 1K\frac{k_1}{k_{-1}} = 1$$
$$K = \frac{k_1}{k_{-1}}$$
Итак, константа равновесия есть соотношение констант скорости прямой и обратной реакций

\subsection*{Кинетика обратимых реакций}

$$aA + bB \Rightarrow cC + dD$$

Скорость обратимой реакции равна разности скоростей прямой и обратной реакции.

$$V_1 = k1 \left[A\right]^a\left[B\right]^b$$
$$V_{-1} = k_{-1}\left[C\right]^c\left[D\right]^d$$
$$V_{sum} = V_1 - V_{-1} = k1 \left[A\right]^a\left[B\right]^b - k_{-1}\left[C\right]^c\left[D\right]^d$$

По мере протекания двусторонней реакции скорость прямой реакции уменьшается, скорость обратной реакции – увеличивается; в некоторый момент времени скорости прямой и обратной реакции становятся равными и концентрации реагентов перестают изменяться. Таким образом, в результате протекания в закрытой системе двусторонней реакции система достигает состояния химического равновесия; при этом константа равновесия будет равна отношению констант скоростей прямой и обратной реакции - см. предыдущий пункт.

\section*{Вопрос №25}

\subsection*{Зависимость скорости реакции от температуры}
\subsection*{Распределение Максвелла-Больцмана}
\subsection*{Уравнение Аррениуса}
\subsection*{Теория активных соударений}
\subsection*{Активированный комплекс}
\subsection*{Понятие о поверхности потенциальной энергии, координате и профиле пути реакции}
\subsection*{Соотношение $E_a$ и $\Delta{H}$}
\subsection*{Cтерический фактор}
\subsection*{Энтропия активации}
\subsection*{Сложные и элементарные реакции, лимитирующая стадия}	



\section*{Вопрос №28}
\subsection*{Катализ}
Катализ -- ускорение реакции под действием веществ, которые сами по себе не претерпевают превращений после завершения процесса.
\subsection*{Катализаторы и ингибиторы химических реакций}
\begin{itemize}
	\item Катализатор -- вещество, ускоряющее реакцию, но не расходующееся в ходе реакции.
	$$ AB + C \rightarrow (ABC) \rightarrow A + BC $$
	Роль катализатора сводится к образованию промежуточного, более реакционноспособного соединения катализатора с одним из реагентов, который затем взаимодействует со вторым реагентом:
	$$ AB + K + C \rightarrow (KAB) + C \rightarrow A + KB + C \rightarrow A + KBC \rightarrow K + A + BC $$
	При этом один процесс сводится к альтернативному пути, представленному двумя реакциями, характеризующимися меньшими энергиями активации $\Rightarrow$ протекающими с большей скоростью.
	\begin{figure}[H]
		\centering
		\includegraphics[scale=0.5]{Kat}
		\label{}
	\end{figure}
	При этом реакция протекает по пути наименьшего сопротивления. В результате ускоряется процесс в целом, а катализатор остается в неизменном виде.
	\item Ингибитор -- вещество, которое замедляет реакцию, но при этом не расходуется. \\
	Ингибитор вынуждает активные частицы переходить в некоторый термодинамически нестабильный продукт по реакции с меньшей энергией активации. Но энергия активации перехода из этого состояния в конечный продукт оказывается очень высокой, и такой процесс оказывается маловероятным. Поэтому промежуточному соединению не остается ничего лучшего, как вновь перейти в исходное состояние.
\end{itemize}
\subsection*{Механизм и кинетика реакций в гомо и гетерогенных системах}
\begin{itemize}
	\item В жидкой фазе гомогенно протекают разнообразные гомолитические реакции распада молекул на радикалы, нуклеофилы, реакции электрофильного замещения, отщепления, перегруппировки, цепные реакции. \\
	Скорость простой гомогенной реакции при постоянном объеме подчиняется закону действующих масс:
	$$ v = k C_A^a C_B^b $$	
	\item Скорость гетерогенной реакции зависит от:
		\begin{itemize}
			\item Скорости подвода реагентов к границе раздела фаз
			\item Скорости реакции на поверхности раздела фаз, которая пропорциональна площади этой поверхности
			\item Скорости отвода продуктов от границы раздела фаз
		\end{itemize}
	Первая и последняя стадии -- диффузионные, вторая -- кинетическая. Та стадия, которая протекает наиболее медленно -- лимитирующая. Именно она определяет скорость реакции в целом. \\
	На скорость реакции влияют: природа реагирующих веществ, концентрации реагентов, температура, наличие катализатора.
\end{itemize}

\subsection*{Общие сведения о кинетике твердофазных процессов}
В твердом теле принципиально существуют две возможности:
\begin{itemize}
	\item Атомы реагентов уже сблизились $\Rightarrow$ взаимодействие практически неизбежно
	\item Атомы реагентов разделены слоем других атомов и молекул $\Rightarrow$ взаимодействие невозможно
\end{itemize}
Скорость твердофазной реакции определяется скоростью доставки реагентов к месту протекания реакции, т.е. скоростью диффузии реагентов через слой продуктов. \\
При попытке осуществить твердофазное взаимодействие двух твердых веществ $A$ и $B$ с образованием твердого продукта $AB$, последний будет образовываться на границе раздела фаз $A$ и $B$. При этом скорость диффузии, согласно закону Фика, будет пропорциональна площади поверхности раздела и градиенту концентраций диффундирующего вещества в слое продукта:
$$ \dfrac{dn}{dt} = -DS \left( \dfrac{dc}{dX} \right)  $$
где $X$ - толщина слоя продукта. 
$$ X^2 = Dt $$
Скорость диффузии в твердом теле весьма мала. Но тем не менее известен ряд твердофазных процессов, протекающих с вполне удовлетворительной скоростью. Чаще всего они происходят при высоких температурах, когда коэффициент диффузии существенно повышается.	
\subsection*{Диффузионно лимитирующиеся процессы}
Диффузионный процесс -- процесс, когда скорость достигает максимума и определяется лишь скоростью роста зародышей. Диффузионными процессами лимитируется скорость окисления многих металлов на воздухе.\\
Так окисление алюминия заканчивается, едва начавшись, ввиду образования тонкой, но плотной оксидной пленки, характеризующейся крайне низким коэффициентом диффузии ионов алюминия и кислорода.
\subsection*{Зародышеобразование}
Зародышеобразование -- процесс образования жизнеспособных центров выделения новой фазы при фазовых переходах первого рода. \\
Зарождение новой фазы происходит при метастабильных состояних исходной системы. \\
Гетерогенное зародышеобразование происходит на посторонних частицах, поверхностях сосудов. \\
При гомогенном зародышеобразовании возникновение устойчивых зародышей происходит в аморфной фазе самопроизвольно вследствие агрегации макромолекул. Наблюдать его в чистом виде практически невозможно.




\end{document}






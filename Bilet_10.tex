\documentclass[14pt,a4paper]{scrartcl}
\renewcommand{\sfdefault}{cmr}

\usepackage[utf8]{inputenc}
\usepackage[english,russian]{babel}

\usepackage{indentfirst}
\usepackage{graphicx}
\usepackage{misccorr}
\usepackage{amsmath}
\usepackage{amssymb}
\usepackage{amsfonts}
\usepackage{icomma}
\usepackage{alltt}
\usepackage{enumitem}
\usepackage{soul}
\usepackage{soulutf8}
\usepackage{graphicx}
\graphicspath{}
\DeclareGraphicsExtensions{.pdf,.png,.jpg}

\begin{document}
	\section*{Вопрос № 10}
	\subsection*{Насыщенный раствор и растворимость}
	Насыщенный раствор – раствор, в котором при данной температуре данное вещество уже больше не растворяется.\\
	Растворимость – способоность вещества образовыввать с другими веществами однородные системы – растворы , в которых вещество находится в виде отдельных атомов, ионов , молекул или частиц. \\
	Растворимость выражается концентрацией растворенное вещества в его насыщенном растворе либо в процентах , либо в весовых или объемных единицах , отнесенных к 100 г или 100 см3.\\
	Растворимость газов в жидкости зависит от температуры и давления. Растворимость жидких и твердых веществ – только от температуры.
	\subsection*{Диаграмма системы соль-вода (на примере $Na_2SO_4 - H_2O$)}
	\subsection*{Зависимость растворимости от температуры}
	\subsection*{Факторы, влияющие на растворимость}
	\subsection*{Криогидратная точка}
	
\end{document}
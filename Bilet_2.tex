\documentclass[14pt,a4paper]{scrartcl}
\renewcommand{\sfdefault}{cmr}

\usepackage[utf8]{inputenc}
\usepackage[english,russian]{babel}

\usepackage{indentfirst}
\usepackage{graphicx}
\usepackage{misccorr}
\usepackage{amsmath}
\usepackage{amssymb}
\usepackage{amsfonts}
\usepackage{icomma}
\usepackage{alltt}
\usepackage{enumitem}
\usepackage{soul}
\usepackage{soulutf8}
\usepackage{graphicx}
\graphicspath{}
\DeclareGraphicsExtensions{.pdf,.png,.jpg}

\begin{document}
	\section*{Вопрос №2}
	\subsection*{Первый закон термодинамики.} 
	Приведу две формулировки:
	\begin{itemize}
		\item В ходе любого процесса изменение внутренней энергии ТС равно разности между количеством сообщенной ей теплоты ($Q$) и совершенной ею работой ($A$):
		$$ (U_2 - U_1) = \Delta{U} = Q - A $$
		\item Сообщенная ТС теплота ($Q$) расходуется на изменение внутренней энергии ТС и совершение работы ($A$):
		$$ Q = \Delta{U} + A $$
	\end{itemize}
	\subsection*{Функция энтальпии.} 	
	Энтальпия ($H$) - функция состояния:
	$$ H = U + pV $$
	Использование энтальпии целесообразно, если ТС совершает  работу по сжатию/расширению, т.к. при $p = const$:
	$$ A = p \Delta{V} = \Delta{(pV)}  $$ 
	Добавим изменение внутренней энергии:
	$$Q = \Delta{U} + A = \Delta{U} + \Delta{(pV)} = \Delta{U + pV} = \Delta{H} $$
	\underline{Энтальпия - мера теплоты процесса, происходящего при постоянном давлении.} 
	
	\subsection*{Тепловой эффект химической реакции при постоянном давлении/объеме/температуре.} 
	\begin{itemize}
		\item При $p = const$: $Q = \Delta{H}$
		\item При $V = const$: $Q = \Delta{U}$, т.к. ТС не совершает работы.
		\item При $T = const$: $Q = A$, т.к. $U$ зависит от $T$ и при $T=const$ : $\Delta{U} = 0$.
	\end{itemize}
	
	\subsection*{Термохимические уравнения.}
	Термохимическое уравнение - уравнение с указанием теплового эффекта. (!) Тепловой эффект отнесен к 1 молю вещества.\\
	В термохимических уравнениях необходимо указывать агрегатные состояния исходных веществ и продуктов реакции. \\
	Существуют экзо- и энотермические реакции:
	\begin{itemize}
		\item Экзотермические реакции происходят с выделением тепла ($\Delta{U}, \Delta{H} < 0$)
		\item Эндотермические реакции происходят с поглощением тепла ($\Delta{U}, \Delta{H} > 0$)
	\end{itemize}
	\subsection*{Теплоты образования и сгорания. Стандартные теплоты и стандартные состояния.}
	\begin{itemize}
		\item Теплота образования -- тепловой эффект реакции образования 1 моля вещества из простых веществ.
		\item Теплота сгорания -- тепловой эффект реакции сгорания одного моля вещества в кислороде до образования оксидов в высшей степени окисления. Теплота сгорания негорючих веществ принимается равной нулю.
		\item Стандартная теплота образования -- тепловой эффект реакции образования одного моля вещества из простых веществ, его составляющих, находящихся в \underline{устойчивых стандартных состояниях.}
		\item Cтандартные состояния -- условно принятые состояния индивидуальных веществ и компонентов растворов при оценке термодинамических величин.  
		Например, для стандартных условий стандартное состояние углерода -- графит, т.к. для стандартных $p$ и $T$ это равновесная модификация углерода.
		Стандартные условия:
		\begin{itemize}
			\item $p = 10^5$ Па
			\item $T = 273,15 K$
		\end{itemize}
	\end{itemize}
	\subsection*{Энергия разрыва химической связи.}
	Энергия химической связи -- мольный прирост энергии вещества при разрушении одной связи определенного типа в каждой молекуле.			
	
	
\end{document}	
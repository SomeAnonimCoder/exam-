\documentclass[14pt,a4paper]{scrartcl}
\renewcommand{\sfdefault}{cmr}

\usepackage[utf8]{inputenc}
\usepackage[english,russian]{babel}

\usepackage{indentfirst}
\usepackage{graphicx}
\usepackage{misccorr}
\usepackage{amsmath}
\usepackage{amssymb}
\usepackage{amsfonts}
\usepackage{icomma}
\usepackage{alltt}
\usepackage{enumitem}
\usepackage{soul}
\usepackage{soulutf8}
\usepackage{graphicx}
\graphicspath{}
\DeclareGraphicsExtensions{.pdf,.png,.jpg}

\begin{document}
	\section*{Вопрос № 8}
	Немножко определений:
	\begin{itemize}
		\item Гомогенная система -- система, внутри которой нет поверхности раздела.
		\item Гетерогенная система -- система, в которой можно выделить части, различающиеся по строению, свойствам и отделенные друг от друга границами раздела.
		\item Фаза ($f$) -- гомогенная часть гетерогенной системы, отделенная от других поверхностью раздела и отличающаяся от них по структуре и свойствам.
		\item Конденсированные системы -- система, в которой газовая фаза не оказывает существенного влияния на фазовые и химические равновесия.
		\item Число независимых компонентов ($k$) -- минимальное число компонентов, из которых можно организовать данную систему. Оно равно общему числу компонентов за вычетом количества связывающих их уравнений.
	\end{itemize}
	\subsection*{Фазовые равновесия}
	\subsection*{Теплоты кипения и плавления}
	\subsection*{Термический анализ}
	\subsection*{Правило фаз Гиббса}
	Число степеней свободы равновесной термодинамической системы превышет разность между числом независимых компонентов и количеством фаз на два.
	$$ f+c = k+2 $$
	Двойка здесь -- влияние температуры и давления. В случае рассмотрения конденсированных систем, правило фаз будет выглядеть так:
	$$ f+c = k+1 $$
	Т.к. давление не оказывает существенного воздействия. \\
	Если нам важно учитывать еще каких-нибудь $n$ параметров (например, электромагнитное поле):
	$$ f+c = k+2 + n $$
	
	\subsection*{Степень свободы, вариантность системы}
	Число степеней свободы (вариантность системы) -- максимальное число переменных, которые можно менять независимо от других, не меняя при этом состояния системы.
	\subsection*{Фазовые диаграммы однокомпонентных систем}
	Фазовая диаграмма -- схема, отражающая области термодинамической стабильности существующих в системе фаз и обозначающая границы раздела между ними.\\
	Фазовое поле -- область фазовой диаграммы, в которой число и качественный состав фаз не меняется. \\
	Для однокомпонентных систем ($f = 1, c = 2$):
	$$ f+c = 3 $$
	\subsection*{Фазовые равновесия на $pT$ - диаграмме}
	\subsection*{$pT$ - диаграмма воды}
	\subsection*{Фазовые поля}
	Фазовое поле -- область фазовой диаграммы, в которой число и качественный состав фаз не меняется.
	\subsection*{Тройная точка}
	Тройная точка -- точка на фазовой диаграмме, в которой сосуществуют три фазы. Для однокомпонентной системы число степеней свободы в тройной точке (из правила фаз Гиббса) равно 0. Равновесие в этой точке -- инвариантное.
	\subsection*{Метастабильные состояния}
	\subsection*{Фазовые переходы первого рода}
	\subsection*{$pT$ - диаграмма серы}
	
\end{document}
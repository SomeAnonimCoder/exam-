\documentclass[14pt,a4paper]{scrartcl}
\renewcommand{\sfdefault}{cmr}

\usepackage[utf8]{inputenc}
\usepackage[english,russian]{babel}

\usepackage{indentfirst}
\usepackage{graphicx}
\usepackage{misccorr}
\usepackage{amsmath}
\usepackage{amssymb}
\usepackage{amsfonts}
\usepackage{icomma}
\usepackage{alltt}
\usepackage{enumitem}
\usepackage{soul}
\usepackage{soulutf8}
\usepackage{graphicx}
\graphicspath{}
\DeclareGraphicsExtensions{.pdf,.png,.jpg}

\begin{document}
	\section*{Вопрос №6}
	
	\subsection*{Термодинамические функции} 
	Термодинамические функции - переменные величины, которые не могут быть непосредственно измерены и зависят от параметров состояния. Наиболее широко применяются следующие термодинамические функции состояния: внутренняя энергия ($U$), энтальпия ($Н$), функция Гельмгольца ($F$), функция Гиббса ($G$), химический потенциал ($\mu_i$)
	\subsection*{Свободная энергия и максимальная работа} 
	Свободная энергия характеризует способность системы превратить часть внутренней энергии в работу. \\
	Максимальная полезная работа равна убыли энергии:
	\[
	A_{max} \leqslant - (\Delta{U + pV - TS})
	\]	
	\subsection*{Свободная энергия Гиббса и Гельмгольца}\
	\begin{itemize}
		\item Свободная энергия Гиббса ($G$) -- та часть внутренней энергии, которую можно превратить в химическую работу \ul{при постоянных давлении и температуре.}
		\[
		G = U - TS + pV = H - TS
		\]
		\[
		\Delta{G} = \Delta{H} - T \Delta{S} - S \Delta{T}
		\]
		что при $T = const$:
		\[
		\Delta{G} = \Delta{H} - T \Delta{S} 
		\]
		\item Свободная энергия Гельмгольца ($F$) -- та часть внутренней энергии, которую можно превратить в химическую работу \ul{при постоянных объеме и температуре.} 
		\[
		F = U -TS
		\]
		\[
		\Delta{F} = \Delta{U} - T \Delta{S} - S \Delta{T}
		\]
		что при $T = const$:
		\[
		\Delta{F} = \Delta{U} - T \Delta{S} 
		\]
		
	\end{itemize}
	\subsection*{Условия самопроизвольного протекания процесса при постоянных $V, T$ и $p, T$} 
	\begin{itemize}
		\item $ V, T = const $ \\
		$ \Delta{F} \leqslant 0 $
		\item $ p, T = const$ \\
		$ \Delta{G} \leqslant 0 $
	\end{itemize}
	
	\subsection*{Химический потенциал} 
	Химический потенциал компонента системы ($\mu$) -- скорость изменения энергии Гиббса при добавлении этого компонента в систему при постоянных давлении, температуре, количествах других веществ. \\
	Для индивидуального вещества -- мольное изменение энергии Гиббса. \\
	При $p, T = const $:
	\[
	\Delta{G} = \Delta{U} - T \Delta{S} + p \Delta{V} = \mu \Delta{n}
	\]
	При $V, T = const $:
	\[
	\Delta{F} = \Delta{U} - T \Delta{S} = \mu \Delta{n}
	\]
	Продифференцировав получим:
	\[
	\mu_i = \left(\dfrac{dG}{dn_i}\right)_{p,T}
	\]
	\[
	G = U + TS + pV \Rightarrow \dfrac{dG}{dp} = V
	\]
	\ul{Пример расчета:} \\
	Для идеального газа 
	\[
	G(p_2) = G (p_1) + \int\limits_{p_1}^{p_2} V dp = G(p_1) + \int\limits_{p_1}^{p_2} nRT \dfrac{p}{dp} = G(p_1) + nRT \ln{\dfrac{p_2}{p_1}}	
	\]
	Для одного моля идеального газа:
	\[
	\mu (p_2) = \mu(p_1) + RT \ln{\dfrac{p_2}{p_1}}	
	\]
	\[
	\mu = \mu^0 + RT \ln{p}
	\]
	где $\mu^0$ -- химический потенциал газа при $p = 1$ атм.
	\subsection*{Активность} 
	Возьмем за меру количества вещества молярную концентрацию ($C$), тогда:
	$$ p=CRT $$
	\[
	\mu (C_2) = \mu(C_1) + RT \ln{\dfrac{C_2}{C_1}}	
	\]
	\[
	\mu = \mu^0 + RT \ln{C}
	\]
	где $\mu^0$ -- химический потенциал раствора при единичной концентрации.
	При переходе от идеальных растворов к реальным:
	\[
	\mu = \mu^0 + RT \ln{a}
	\]
	где $a = \gamma c$ -- активность. 
	\subsection*{Термодинамические расчеты} 
	
\end{document}
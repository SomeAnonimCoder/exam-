\documentclass[14pt,a4paper]{scrartcl}
\renewcommand{\sfdefault}{cmr}

\usepackage[utf8]{inputenc}
\usepackage[english,russian]{babel}

\usepackage{indentfirst}
\usepackage{graphicx}
\usepackage{misccorr}
\usepackage{amsmath}
\usepackage{amssymb}
\usepackage{amsfonts}
\usepackage{icomma}
\usepackage{alltt}
\usepackage{enumitem}
\usepackage{soul}
\usepackage{soulutf8}
\usepackage{graphicx}
\graphicspath{}
\DeclareGraphicsExtensions{.pdf,.png,.jpg}

\begin{document}
	\section*{Вопрос №1}
	
	\subsection*{Основные задачи химической термодинамики.} 
	Термодинамика изучает количественные соотношения между
	теплотой, работой и различными формами энергии, в том числе
	и химической. \\
	Химическая термодинамика изучает превращение энергии
	химических реакций в теплоту и работу. \\
	Основные задачи:
	\begin{itemize}
		\item Определение условий реализации химических процессов. Вычисление тепловых эффектор химических реакций.
		\item Поиск пределов устойчивости исследуемых веществ при заданных условиях.
		\item Избрание оптимального режима проведения процесса.	
	\end{itemize}
	
	\subsection*{Термодинамические параметры.} 
	\begin{itemize}
		\item Энтальпия
		\item Энтропия
		\item Энергия Гиббса
		\item Теплоемкость
	\end{itemize}
	
	\subsection*{Классификация систем.} 
	Под термодинамической системой (ТС) понимается некоторая часть пространства со всеми включенными в нее компонентами. Всякая ТС должна быть ограничена реальной или вооброжаемой границей - поверхностью раздела. Через поверхность может осуществляться обмен веществом или энергией.
	\begin{itemize}
		\item Открытая система -- ТС, в которой разрешены все типы обмена.
		\item Замкнутая система -- ТС, в которой разрешен только обмен энергией.
		\item Адиабатическая система -- ТС, в которой разрешен только обмен веществом.
		\item Изолированная система -- ТС, в которой запрещены любые обмены.
	\end{itemize}
	
	\subsection*{Экстенсивные и интенсивные термодинамические параметры.} 
	Для полного описания системы мы выбираем минимальный набор параметров - независимых переменных (который зависит от условий эксперимента). Назовем их параметрами состояния.
	\begin{itemize}
		\item Экстенсивные параметры состояния (ЭПС) - параметры, определяемые количеством вещества в системе: $ V, m, l, S, q  $. ЭПС аддитивны.
		\item Интенсивные параметры состояния (ИПС) - параметры, которые не зависят от количества вещества и могут быть измерены лишь опосредовано через ЭПС. Примеры: $p, T $
	\end{itemize}
	Любые виды работы ($A$) могут быть охарактеризованы так: (где Y - ИПС, а Х - ЭПС):
	$$ dA = Y dX  $$ 
	Например:
	$$ dA = P dV  $$
	\subsection*{Внутренняя энергия.} 
	Каждое вещество характеризуется потенциальным запасом энергии. Она включает в себя все виды энергии движения и взаимодействия частиц. Эта величина - внутренняя энергия ($U$). Она зависит только от термодинамических параметров ТС и не зависит от путей достижения конкретного состояния ТС, а значит $U$ - функция состояния. \\
	Мы никогда не работаем с абсолютными значениями $U$, а лишь с ее изменениями $\Delta{U}$.
\end{document}




\documentclass[14pt,a4paper]{scrartcl}
\renewcommand{\sfdefault}{cmr}

\usepackage[utf8]{inputenc}
\usepackage[english,russian]{babel}

\usepackage{indentfirst}
\usepackage{graphicx}
\usepackage{misccorr}
\usepackage{amsmath}
\usepackage{amssymb}
\usepackage{amsfonts}
\usepackage{icomma}
\usepackage{alltt}
\usepackage{enumitem}
\usepackage{soul}
\usepackage{soulutf8}
\usepackage{graphicx}
\graphicspath{}
\DeclareGraphicsExtensions{.pdf,.png,.jpg}

\begin{document}
	\section*{Вопрос №5}
	
	\subsection*{Энтропия идеального кристалла} 
	Постулат Планка: \\
	Энтропия идеального кристала при $0K$ равна $0$. \\
	В процессе охлаждения снижается амплитуда колебаний атомов в кристаллической решетке, снижается вероятность ее изменения $\Rightarrow$ понижается степень свободы $\Rightarrow$ понижается энтропия кристалла в целом. В пределе выполняется постулат Планка. 
	\subsection*{Энтропия идеального газа} 
	Пусть 1 моль газа нагрели от $T_1$ до $T_2$, газ расширился от $V_1$ до $V_2$. Сообщенное газу $Q$ на каждом малом участке уходит на увеличение внутренней энергии $C_V dT$ и на работу по расширению $ RT \frac{dV}{V} $, тогда суммарное изменение энтропии:
	\[
	\Delta{S} = \int\limits_{T_1}^{T_2} C_V \dfrac{dT}{T} + \int\limits_{T_1}^{T_2} R \dfrac{dV}{V} = 
	C_V \ ln{\frac{T_2}{T_1}} + R \ln{\frac{V_2}{V_1}} 
	\]
	
	\subsection*{Изменение энтропии при постоянном объеме и постоянном давлении} 
	\begin{itemize}
		\item При постоянном объеме работа равна 0:
		$$ 	\Delta{S} = \int\limits_{T_1}^{T_2} C_V \dfrac{dT}{T} = C_V \ ln{\frac{T_2}{T_1}} $$ 
		\item При постоянном давлении:
		$$ \Delta{Q} = \Delta{H} = C_p dT $$
		$$ \Delta{S} = \int\limits_{T_1}^{T_2} C_p \dfrac{dT}{T} = C_p \ ln{\frac{T_2}{T_1}}  $$
	\end{itemize}
	\subsection*{Изменение энтропии в необратимых процессах} 
	Т.к. энтропия -- функция состояния, то ее изменение будет зависеть только от начального и конечного состояния системы и одинаково для всех путей перехода между этими состояниями, включая обратимый. Поэтому в случае неравновесного процесса его следует разбить на равновесные. \\ 
	Пример: Неравновесное расширение газа против меньшего давления с нагреванием системы = равновесное расширение + нагревание при постоянном объеме.
	
	
\end{document}
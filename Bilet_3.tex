\documentclass[14pt,a4paper]{scrartcl}
\renewcommand{\sfdefault}{cmr}

\usepackage[utf8]{inputenc}
\usepackage[english,russian]{babel}

\usepackage{indentfirst}
\usepackage{graphicx}
\usepackage{misccorr}
\usepackage{amsmath}
\usepackage{amssymb}
\usepackage{amsfonts}
\usepackage{icomma}
\usepackage{alltt}
\usepackage{enumitem}
\usepackage{soul}
\usepackage{soulutf8}
\usepackage{graphicx}
\graphicspath{}
\DeclareGraphicsExtensions{.pdf,.png,.jpg}

\begin{document}
\section*{Вопрос №3}	

\subsection*{Расчет тепловых эффектов реакций по теплотам образования, сгорания и разрыва химических связей.}
\begin{itemize}
	\item По теплотам образования: 
	$$\Delta{H_{p}^{o}} = \sum \Delta{H_{f}^o} \text{(продукты)} - \sum \Delta{H_{f}^o} \text{(реагенты)}  $$
	\item По теплотам сгорания: \\
	Аналогично первому методу, только необходимо брать энтальпию сгорания.
	\item По энергии химических связей:
	Зная состав и химическую формулу (со всеми связями между атомами вещества) можно оценить, какова его энергия формирования. Точные значения энергий конкретных связей -- справочная информация, но известно, что по силе взаимодействия:
	$$ -  <  =  <  \equiv $$
\end{itemize}
\subsection*{Закон Гесса и термохимия.} 
Закон Гесса: \\
Тепловой эффект химического процесса зависит только от природы и состояний исходных веществ и продуктов, но не от пути его осуществления, в том числе от выбора системы реакций и состояния промежуточных продуктов. \\
Пример расчетов: \\
$$C\text{(графит)} + \frac{1}{2}O_2\text{(г)} = CO\text{(г)}: \Delta{H_1}$$  
$$C\text{(графит)} + \frac{1}{2}O_2\text{(г)} = CO_2\text{(г)}: \Delta{H_2} = -393,5 \text{кДж/моль}$$  			
$$CO\text{(г)} + \frac{1}{2}O_2\text{(г)} = CO_2\text{(г)}: \Delta{H_3} = -110,5 \text{кДж/моль} $$  
Отсюда:
$$\Delta{H_1} = \Delta{H_2} - \Delta{H_3} $$
\subsection*{Теплоемкость.} 		
Истинная теплоемкость:
$$C_T = \dfrac{dQ}{dT} $$
Т.е. отношение теплоты ($dQ$), которая требуется, чтобы нагреть ТС на $dT$ к изменению температуры ($dT$). Тогда чтобы нагреть ТС от температуры $T_1$ до $T_2$:
$$ Q = \int\limits_{T_1}^{T_2} C_T dT $$
\subsection*{Теплоемкость идеального газа.} 
Следует различать теплоемкость при постоянном давлении ($C_p$) и теплоемкость при постоянном объеме ($C_V$). Т.к. при $p=const$ теплота также расходуется на работу расширения. Поэтому:
$$C_V = \dfrac{\frac{dU}{dT}}{n} $$
$$C_p = \dfrac{\frac{dH}{dT}}{n} $$
где $n$ - количество вещества. Для твердых и жидких веществ $C_p \approx C_V$. Для 1 моля идеального газа:
$$C_p = \dfrac{dH}{dT} = \dfrac{d(U+pV)}{dT} = \dfrac{d(U+RT)}{dT} = \dfrac{dU}{dT} + R = C_V + R $$
\subsection*{Теплоемкость одноатомного и многоатомных газов.} 
Из МКТ известно, что:
$$E = \dfrac{3}{2} kT \Rightarrow \Delta{U} = \dfrac{3}{2} RT $$
Из теплоемкости идеального газа следует, что для одноатомного газа $C_V = \frac{3}{2} R $ и $C_p = \frac{5}{2}R$. На каждую поступательную степень свободы - $\frac{1}{2} R$. Столько же приходится и на вращательные. Тогда если в молекуле газа $N$ атомов $\Rightarrow 3N$ степеней свободы:
$$ C_V = \dfrac{3N}{2} R $$
$$ C_p = \dfrac{3N+2}{2} R $$
Однако эти соотношения работают при больших температурах, при низких нужно учитывать вклад только вращательных степеней свободы, которых для линейных молекул 2, а для нелинейных -- 3.
\subsection*{Зависимость теплоемкости и энтальпии вещества от температуры.} 
\subsection*{Общие понятия о фазовых переходах} 		
\subsection*{Зависимость тепловых эффектов химических реакций от температуры.} 		
\subsection*{Уравнение Кирхгоффа.} 	
\end{document}
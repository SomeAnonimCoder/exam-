\documentclass[14pt,a4paper]{scrartcl}
\renewcommand{\sfdefault}{cmr}

\usepackage[utf8]{inputenc}
\usepackage[english,russian]{babel}

\usepackage{indentfirst}
\usepackage{graphicx}
\usepackage{misccorr}
\usepackage{amsmath}
\usepackage{amssymb}
\usepackage{amsfonts}
\usepackage{icomma}
\usepackage{alltt}
\usepackage{enumitem}
\usepackage{soul}
\usepackage{soulutf8}
\usepackage{graphicx}
\graphicspath{}
\DeclareGraphicsExtensions{.pdf,.png,.jpg}

\begin{document}
	\section*{Вопрос №4}
	
	\subsection*{Второй закон термодинамики} 
	Существуют некоторые процессы, не противоречащие первому закону термодинамики, которые самопроизвольно протекать не могут.\\ \\
	Процессы, которые не могут протекать самопроизвольно -- отрицательные. Отрицательный процесс не может являться единственным результатом действия. \\
	Постулаты Клаузиса и Томсона:
	\begin{itemize}
		\item Теплота не может самопроизвольно переходить от холодного тела к горячему.
		\item Теплота более холодного из участвующих в процессе тел не может служить источником работы.
	\end{itemize}
	\subsection*{Обратимые и необратимые процессы} 
	\begin{itemize}
		\item Обратимый процесс -- процесс, при котором в любой фазе превращения все части рассматриваемой системы находятся в равновесии друг с другом и с внешним окружением. \\
		При обратимом процессе: 
		$$ \Delta{U}, A = const  \Rightarrow Q = const  $$
		Обратимый процесс можно осуществить единственным образом, поэтому $Q$ такого процесса  -- функция состояния.
		\item Необратимый процесс -- процесс, который нельзя провести в обратном направлении так, чтобы не произошло изменений в окружающей среде.
	\end{itemize}
	\subsection*{Энтропия} 
	Энтропия -- мера разупорядоченности системы.
	\begin{itemize}
		\item При обратимом процессе:
		$$\Delta{S} = S_2 - S_1 = \dfrac{Q}{T} $$ 
		\item При необратимом процессе:
		$$ dS = \dfrac{dQ}{T} + \dfrac{dI}{T} $$
		где $dI$ -- поток энергии во внешнее пространство (из-за того, что всегда $A_{HO} < A_O$, оставшаяся энергия выделяется в виде тепла), поэтому всегда:
		$$ \Delta{S} > \dfrac{Q}{T} $$
	\end{itemize}
	\subsection*{Направление самопроизвольного процесса в изолированной системе} 
	В изолированной системе самопроизвольно могут протекать только процессы, сопровождающиеся положительным изменением энтропии.
	\subsection*{Статистическая природа второго закона термодинамики} 
	\begin{itemize}
		\item Термодинамическая вероятность ($W$) -- число микросостояний, которыми мы можем реализовать данное состояние системы.
		
	\end{itemize}
	Система должна стремиться к наиболее вероятному состоянию, поэтому:
	$$ S = k \ln{W} $$
	Пример использования: \\
	При увеличении объема одного моля идеального газа в $\frac{V_2}{V_1}$ раза вероятность возрастает в $(\frac{V_2}{V_1})^N$ раз, тогда получим:
	$$ \Delta{S} = k \ln{(\frac{V_2}{V_1})^N} = k N \ln{\frac{V_2}{V_1}} = R \ln{\frac{V_2}{V_1}} $$ 
\end{document}
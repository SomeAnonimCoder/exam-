\documentclass[14pt,a4paper]{scrartcl}
\renewcommand{\sfdefault}{cmr}

\usepackage[utf8]{inputenc}
\usepackage[english,russian]{babel}

\usepackage{indentfirst}
\usepackage{graphicx}
\usepackage{misccorr}
\usepackage{amsmath}
\usepackage{amssymb}
\usepackage{amsfonts}
\usepackage{icomma}
\usepackage{alltt}
\usepackage{enumitem}
\usepackage{soul}
\usepackage{soulutf8}
\usepackage{graphicx}
\graphicspath{}
\DeclareGraphicsExtensions{.pdf,.png,.jpg}

\begin{document}
	\section*{Вопрос №7}
	\subsection*{Обратимость химических реакций}
	Обратимые химические реакции -- реакции, которые одновременно идут в двух взаимно противоположных направлениях. \\
	\ul{Не путать с обратимыми процессами:} обратимая реакция не обязательно должна протекать в равновесных условиях.
	\subsection*{Химическое равновесие}
	Химическое равновесие -- состояние системы, при котором скорость прямой реакции равна скорости обратной. \\
	При состоянии химического равновесия реакция не прекращается, а обе реакции идут с равными скоростями. \\
	Математическое выражение условия равновесия:
	\[
	\Delta{H} = T \Delta{S} \Rightarrow \Delta{G} = 0
	\]
	\subsection*{Условия химического равновесия в гомо- и гетерогенных системах}
	Для гомогенных процессов:
	\begin{itemize}
		\item Равновесие в реакция, протекающих в газовой фазе: \\
		Пусть есть реакция:
		$$ aA + bB = cC +  dD $$
		Тогда условие равновесия:
		$$ \Delta{G} = cG_c + dG_D - aG_A - bG_b $$
		Выразим химические потенциалы через стандартые с учетом давления (т.к. речь про газы):
		$$ \mu_i = \mu^o_i + RT \ln{p} $$
		Тогда:
		$$ \Delta{G} = c\mu_C + d\mu_D - a\mu_A - b\mu_B = $$
		$$ =c\mu^o_C + d\mu^o_D - a\mu^o_A - b^o\mu_B + RT \ln{(c\ln{p_C} + d\ln{p_D} - a\ln{p_A} - b\ln{p_B})} = $$
		$$ = \Delta{G^o} + RT \ln{\left[ \dfrac{(p_C)^c(p_D)^d}{(p_A)^a(p_B)^b}\right]} $$
		В условиях равновесия величина в квадратных скобках -- константа равновесия. Поскольку при равновесии $\Delta{G} = 0$:
		$$ \ln{\left[ \dfrac{(p_C)^c(p_D)^d}{(p_A)^a(p_B)^b}\right]} = \ln{K_p} = - \dfrac{\Delta{G^o}}{RT}  $$
		\item Равновесие в реакциях, протекающих в растворах:\\
		Расчет аналогичен газам, только вместо давления следует рассматривать зависимость потенциала от концентрации или активности.
	\end{itemize}
	Для гетерогенных процессов: \\
	Пусть есть процесс разложения карбоната кальция:
	$$ CaCO_3 = CaO + CO_2 $$
	Для равновесия при атмосферном давлении:
	$$ \Delta{G} = \mu_{CO_2} + \mu_{CaO} - \mu_{CaCO_3} = \mu_{CO_2}^o + RT \ln{p_{CO_2}} + \mu_{CaO}^o - \mu_{CaCO_3}^o = $$
	$$ = \Delta{G^o} + \ln{p_{CO_2}} $$
	Т.е. константой равновесия данного процесса будет парциальное давление $CO_2$. \\
	В общем случае: если гетерогенный процесс происходит с участием газов, нужно учитывать лишь  их парциальное давление. Если процесс в растворах с выделением/расстворением осадка -- аналогично расчетам гомогенных реакций в расстворах.
	\subsection*{Глубина протекания процессов}
	
	\subsection*{Степень превращения}
	\subsection*{Константа равновесия}
	Состояние химического равновесия количественно характеризуется константой равновесия, представляющей собой отношение констант прямой ($K_1$) и обратной ($K_2$) реакций. Таким образом, для реакции
	$$ aA + bB = cC +  dD $$
	Константа равновесия:
	$$ K_p = \frac{K_1}{K_2} = \dfrac{\left[ C\right]^c \left[ D\right]^d }{\left[ A\right]^a \left[ B\right]^b} $$
	\subsection*{Факторы, влияющие на величину константы равновесия}
	Константа равновесия не зависит от давления и концентраций, но зависит от температуры.
	\subsection*{Смещение положения равновесия, принцип подвижного равновесия Ле Шателье - Брауна}
	Если на систему, находящуюся в состоянии подвижного равновесия, оказывается внешнее воздействие, то положение равновесия смещается в сторону, противодействующую этому воздействию. \\
	Пример: \\
	\begin{itemize}
		\item $ \Delta{H} > 0, \Delta{n} = 0 $ 
		$$ H_2 + I_2 = 2HI $$
		При увеличении температуры равновесие смещается в сторону продуктов. \\
		Давление никак не влияет на равновесие, т.к. одинаковое количество молекул газа и слева, и справа.
		\item $ \Delta{H} < 0, \Delta{n} < 0 $
		$$ 3H_2 + N_2 = 2 NH_3  $$
		При увеличении температуры равновесие смещается в сторону реагентов. \\
		При увеличение давления -- в сторону продуктов, т.к. справа 2 молекулы газа, а слева -- 4.
		\item $ \Delta{H} > 0, \Delta{n} > 0 $
		$$ N_2O_4 = 2NO_2 $$
		При увеличении температуры равновесие смещается в сторону продуктов. \\
		При увеличении давления -- в сторону реагентов.  	
	\end{itemize}
	
	\subsection*{Стандартная свободная энергия}
	Это разность между свободной энергией исходных веществ и свободной энергией продуктов реакции.
	
\end{document}